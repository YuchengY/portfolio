% !TeX program = pdfLaTeX
\documentclass[12pt]{article}
\usepackage{amsmath}
\usepackage{graphicx,psfrag,epsf}
\usepackage{enumerate}
\usepackage{natbib}
\usepackage{textcomp}
\usepackage[hyphens]{url} % not crucial - just used below for the URL
\usepackage{hyperref}
\providecommand{\tightlist}{%
  \setlength{\itemsep}{0pt}\setlength{\parskip}{0pt}}

%\pdfminorversion=4
% NOTE: To produce blinded version, replace "0" with "1" below.
\newcommand{\blind}{0}

% DON'T change margins - should be 1 inch all around.
\addtolength{\oddsidemargin}{-.5in}%
\addtolength{\evensidemargin}{-.5in}%
\addtolength{\textwidth}{1in}%
\addtolength{\textheight}{1.3in}%
\addtolength{\topmargin}{-.8in}%

%% load any required packages here



% Pandoc citation processing

\usepackage{setspace}
\usepackage{booktabs}
\usepackage{longtable}
\usepackage{array}
\usepackage{multirow}
\usepackage{wrapfig}
\usepackage{float}
\usepackage{colortbl}
\usepackage{pdflscape}
\usepackage{tabu}
\usepackage{threeparttable}
\usepackage{threeparttablex}
\usepackage[normalem]{ulem}
\usepackage{makecell}
\usepackage{xcolor}

\begin{document}


\def\spacingset#1{\renewcommand{\baselinestretch}%
{#1}\small\normalsize} \spacingset{1}


%%%%%%%%%%%%%%%%%%%%%%%%%%%%%%%%%%%%%%%%%%%%%%%%%%%%%%%%%%%%%%%%%%%%%%%%%%%%%%

\if0\blind
{
  \title{\bf Exploring the Finite-sample Behavior of Residual
Diagnostics for Linear Mixed-effects Models}

  \author{
        Yicheng Shen
~~\href{mailto:sheny2@carleton.edu}{\nolinkurl{sheny2@carleton.edu}} \\
    Department of Mathematics and Statistics, Carleton College\\
     and \\     Yucheng Yang
~~\href{mailto:yangy2@carleton.edu}{\nolinkurl{yangy2@carleton.edu}} \thanks{This
research was supported by the Towsley Endowment for the Sciences.} \\
    Department of Mathematics and Statistics, Carleton College\\
      }
  \maketitle
} \fi

\if1\blind
{
  \bigskip
  \bigskip
  \bigskip
  \begin{center}
    {\LARGE\bf Exploring the Finite-sample Behavior of Residual
Diagnostics for Linear Mixed-effects Models}
  \end{center}
  \medskip
} \fi

\bigskip
\begin{abstract}
Understanding the residual behaviors has always been a key challenge for
analysts who need to examine the validity of fitted models. In the
framework of linear mixed-effects (LME) models, we employ
simulation-based methods that provide an abundant set of artificial data
and models with and without deficiencies to explore the finite-sample
behavior of residual diagnostics. The results of our simulations point
to the intertwined nature of LME model assumptions: a single or a pair
of misspecifications often lead to structures in residual plots that
could be flagged as problematic to other assumptions. The way in which
the hierarchical data was composed, namely cluster sizes, residual
variances, and longitudinal settings, are all influential to the
residual diagnostics. These findings have major implications for the LME
model checking procedures using residual analysis and can serve as
guidelines for interpreting residual diagnostic plots and test
statistics.\\
\hspace*{0.333em}
\end{abstract}

\noindent%
{\it Keywords:} linear mixed-effects models, simulation study,
distributional assumptions, hierarchical linear models, residual
diagnostics
\vfill

\newpage
\spacingset{1.45} % DON'T change the spacing!

\onehalfspacing

\section{Introduction}

Linear mixed-effects (LME) models, which account for the interdependence
of observations that arise from a hierarchical data structure, allow for
the analysis of clustered data in a wide range of settings, such as
agricultural and ecological experiments, longitudinal studies, and
educational assessments. As with all model-based methods, the validity
of model assumptions should always be carefully checked to ensure that
the fitted model adequately represents the realities that exist in the
data. While the existing diagnostic procedures \citep{Singer2017-sd}
appear to work well in many situations \citep{Schutzenmeister2012},
evidence suggests that these conventional tools are inadequate if
certain structures appear in the data \citep{Loy2015-vl}. For example,
the structure of residual plots under unbalanced group sizes can induce
unusual patterns in residual plots when the fitted model is in fact
adequate for analysis\citep{Morrell2000-ut}.

In this paper, we present a simulation study exploring the finite-sample
behaviors of residual diagnostics for LME models. In Section 2, we
discuss the backgrounds of LME model specification, model assumptions
and proposed diagnostics. In Section 3, we outline our simulation study,
from constructing data sets and fitting models, to extracting residuals
and conducting analysis. The results of our study are presented in
Section 4, where we explore diagnostic tools for detecting assumption
violations under ten different scenarios along with interpretations of
residual behaviors. Patterns and findings are generalized and discussed
in Section 5.

The results of our study indicate that most LME model assumptions for
residuals are entwined and interdependent: when one or more
distributional assumption is misspecified in the data, practitioners are
likely to observe patterns that suggest the breakdowns of other model
assumptions in residual plots. The distribution of group sizes and the
relative magnitude of the variability of error terms and random effects
could also influence the frequency of violations seen in residual
diagnostics: having unbalanced cluster sizes can exaggerate patterns of
non-normality, or mitigate the heteroscedasticity of residual
quantities, whereas the residual variance plays an important role in
inducing problematic structures in residual plots. Notable exceptions
are the violation of p normal assumption of random effects and the
bimodality of errors, both of which have milder impacts on the validity
of LME model assumptions. These results have far-reaching implications
for LME model checking using residual plots and could serve as future
guidelines for properly evaluating the diagnostic plots and test
statistics.

\section{Linear mixed-effects models}

With the advancement of sophisticated data collection and processing
tools, multiple-level and non-independent data structures have become
the norm in many domains, particularly the biological and social
sciences \citep{bolker2008ecological, raudenbush2002hierarchical}.
Neglecting to take this complex correlation into account could lead to
underestimating standard errors of coefficients, overstating the
significance of predictors and generating biased estimates
\citep{gurka2011avoiding, Roback2021}. LME models allow us to
appropriately model these hierarchical structures in clustered data.

\subsection{Model Specification}

A standard LME model for \(n\) observations nested in \(m\) groups is
given by \begin{equation}
\bf{y_i} = \bf{X_i} \boldsymbol{\beta} + \bf{Z_ib_i} + \boldsymbol{\epsilon_i} \ \ \ \ \ \  \text{for  i = 1, 2, 3, ... , m} 
\end{equation} where \(\bf{y_i}\) is a \(n \times 1\) response vector;
\(\bf{X_i}\) is a \(n \times p\) matrix representing \(p\) predictors;
\(\boldsymbol{\beta}\) is a \(p\times 1\) vector of fixed-effect
coefficients. \(\bf{Z_ib_i + \epsilon_i}\) stands for the random
components in the data structure: \(\bf{Z_i}\) is a \(n \times q\) model
matrix for the \(q\) number of random effects in \(\bf{b_i}\), and
\(\boldsymbol{\epsilon_i}\) is a \(n \times 1\) vector of within-group
measurement errors. Both random effects, \(\bf{b_i}\), and residual
errors, \(\boldsymbol{\epsilon_i}\), are independent random variables,
and we assume that
\begin{equation} \boldsymbol{b_1, ... b_i} \overset{iid} \sim N_q(\boldsymbol{0}, \sigma^2 \boldsymbol{G}) \ \ \text{and} \ \ \boldsymbol{\epsilon_i} \overset{iid}\sim N_n(0,\sigma^2 \boldsymbol{R_i})  \ \ \ \ \ \text{for  i = 1, 2, 3, ... , m} \end{equation}
where \(\bf{G}\) and \(\bf{R_i}\) are \(q\) and \(n\) dimensional
positive definite matrices, with elements expressed as functions of a
vector of covariance parameters \citep{laird1982random}.

LME models are featured by incorporating both fixed and random effects
within a hierarchical structure \citep{bates1998computational}. The
fixed effects in LME models are regression coefficients that correspond
to fixed quantities of interest. The random effects in LME models
represent the correlation of intercepts and slopes in hierarchical data,
adjusting for non-independence between observations and group-specific
profiles.

To model both the mean and covariance structures, estimates must be
obtained for both fixed effects and variance components. One possible
approach is to maximize the likelihood function
\citep{mcculloch1997maximum}. Due to the bias of maximum likelihood (ML)
estimates for the variance components, restricted maximum likelihood
(REML) has been more widely adopted \citep{gilmour1995average}.
Regardless of the procedures used, reliable estimation must be based
upon models that correctly depict the structure of data.
\citet{verbeke1996linear} illustrated that wrongly assuming normality of
random effects leads to incorrect estimates of random effects
coefficients. The misspecification of random effects can cause
substantial bias of estimations of regression coefficients from LME
models \citep{heagerty2001misspecified, Hui2021}.

\subsection{Model Diagnostics}

Fitting correctly specified multilevel models that converge is a crucial
step of formulating a solid and reliable inferences for statisticians
\citep{Gelman_hill_2006}. A broad set of model checking guidelines and
techniques have been developed and implemented in statistical software
over the years for inspecting the appropriateness of model assumptions,
including residual normality, linearity and homoscedasticity
\citep{pinheiro2000linear, Cheng2010-ec, Bates2015}.

Residual analysis for LME models requires three types of residual
quantities generated from the fitted models
\citep{santos2007residual, haslett1998residuals}:

\begin{itemize}
\item
  The \emph{marginal residuals} estimate the model's random components,
  \(\boldsymbol{Z_i b_i + \epsilon_{i}}\). They are defined by
  \(\boldsymbol{\hat \delta_i = y_i - \hat{E}(y_i) = Z_i\hat{b_i} + \hat \epsilon_{i}}\).
  Here \(\bf{\hat{E}(y_i)}\) is the best linear unbiased estimates
  (BLUEs) using all data.
\item
  The \emph{conditional residuals} are the residual deviations that
  estimate the model's error term, \(\boldsymbol\epsilon_i\). They are
  defined by
  \(\boldsymbol{\hat \epsilon_i = y_i - \hat{E}(y_i|b_i) = y_i - X_i\hat{\beta_i} - Z_i\hat{b_i}}\).
\item
  The \emph{predicted random effects residuals}, \(\bf{Z_i \hat b_i}\),
  estimate the model's random effects, \(\bf{Z_i b_i}\). They are the
  best linear unbiased predictors (BLUPs) given by
  \(\boldsymbol{\hat E(y_i|b_i)-\hat E(y_i)}\).
\end{itemize}

In our simulation study, we standardized residuals for model
diagnostics. We employed Cholesky residuals as standardized marginal
residuals \citep{houseman2004cholesky}. The original marginal covariance
matrix and its Cholesky decomposition as \(L(y)\), which is defined as
\(V(y)^{-1} = L(y)L(y)^\top\). Then the Cholesky residual is defined as
\(\delta_i^* = L(y)^\top \delta_i\). \citet{santos2007residual}
suggested standardizing the raw conditional residuals, \(\hat {e_i}\),
by its \(\hat \sigma\). The standardized conditional residuals are thus
given by
\(\hat {e_i}^* = \frac{\hat {e_i}}{\hat \sigma \sqrt{\hat p_{kk}}}\),
where the elements \(\hat p_{kk}\) is the functions of the joint
leverage of the fixed and random effects.

\citet{Snijders2008-dm} discussed the diagnostics for two-level
hierarchical linear mixed-effects models, highlighting the important
role of residual analysis and deriving various properties of the
residuals at both the individual and cluster levels. They suggest that
the main use of the marginal residuals is to investigate the
specification of within-cluster linearity and homoscedasticity. The
presence of outliers and potential effects of omitted variables can also
be studied using the marginal residuals. The conditional and random
effects residuals, on the other hand, can be plotted to check their
normality assumptions.

\citet{Singer2017-sd} further generalized the usage of visual diagnostic
tools: plotting standardized marginal and conditional residuals versus
the explanatory variables or fitted values is the standard approach for
LME model diagnostics of linearity, normality, independence, and
heteroscedasticity of errors. The normality assumptions of random
effects are measured by the Mahalanobis Distance (MD)
\citep{mahalanobis1936generalized, mclachlan1999mahalanobis}: if random
effects follow multivariate normal distribution, their MD should follow
a chi-squared distribution with \(q\) degrees of freedom, where \(q\)
denotes the number of random effects.

However, residual methods that work well in linear regression with i.i.d
errors have been less successful in LME model diagnostics because the
empirical distribution of residuals does not necessarily converge to the
true distribution of the errors \citep{jiang1998asymptotic}. In
practice, conventional residual plots and tests for model validation
perform poorly in finite sample situations \citep{Loy2017-fo}. The
assessment of the normality assumption by interpreting conventional
quantile--quantile (QQ) plots suffers from inflated Type I error rates
\citep{Loy2015-vl}. \citet{Schutzenmeister2012} argued that another
problem for QQ plots is the difficulty of assessing whether the
curvature in the plotted residuals is indicative of a departure from
normality or whether there are possible outliers. Moreover, if a cluster
has exactly one observation, the plot of any estimated random effect
against any other estimated random effect will fall on a straight line
\citep{Morrell2000-ut}, requiring researchers to take extra cautions
when evaluating heteroscedasticity in the set of repeated-measures or
longitudinal data.

\section{Simulation Study}

To address the inadequacy of the scope of existing literature on this
topic, we propose a simulation-based approach to study residual
diagnostic of LME models under finite samples.

\subsection{Baseline Data Generation}

In this section, we introduce how we simulated data from a model with
two fixed effects and two random effects. Fitting these data with
properly specified models serves as the baseline for residual
performance for two-level LMEs.
\begin{equation} Y_{i,j} =  \underbrace{\beta_0 + \beta_1 X_{i,j} + \beta_2 Z_{i,j}}_\text{Fixed Effects} + \underbrace{u_i + v_i X_{i,j}}_\text{Random Effects}  + \underbrace{\epsilon_{i,j}}_\text{Error} \ \ \ \text{for} \ i \in \{ 1:N \} \ \& \ j \in \{ 1:n \} \end{equation}
where \(\epsilon_{i,j}\overset{iid}\sim N(0,\sigma^2)\) and
\(\begin{bmatrix} u_i\\v_i \end{bmatrix} \sim \text{MVNorm} \left( \begin{bmatrix} 0 \\ 0 \end{bmatrix},\begin{bmatrix} \sigma_1^2&\rho_{1,2}\sigma_1\sigma_2\\ \rho_{1,2}\sigma_1\sigma_2&\sigma_2^2\end{bmatrix} \right)\)

~

To account for the influences of unbalanced cluster sizes, coefficients
and variability of fixed effects, random effects and errors, we
considered the following settings when simulating data.

\begin{itemize}
\item
  \textbf{\emph{Generate Clusters}}: We have three kind of cluster
  distribution settings:

  \begin{itemize}
  \tightlist
  \item
    Same size group setting: 25 observations per group, 50 groups;
  \item
    Balanced group setting: 20-30 observations per group, 50 groups;
  \item
    Unbalanced group setting: 2-50 observations per group, 50 groups.
  \end{itemize}
\item
  \textbf{\emph{Simulate Predictors}}: The two predictors, denoted as
  \(X_{i,j}\) and \(Z_{i,j}\), were drawn from independent \(N(0,1)\)
  distributions. The intercept, \(\beta_0\), was set as 5 and both
  slopes, \(\beta_1\) and \(\beta_2\), were set to 1.
\item
  \textbf{\emph{Simulate Random Effects}}: We included two random
  effects, a random intercept and a random slope for \(X_{i,j}\). The
  random intercept and slope are assumed to be normally distributed and
  correlated, with mean zero and correlation coefficient \(\rho = 0.5\).
  For high variance random effects, we set \(\sigma_1 = \sigma_2 = 5\);
  for low variance random effects, both standard deviations were set to
  1.
\item
  \textbf{\emph{Simulate Error Terms}}: For large error variance, we set
  \(\epsilon_{i,j}\overset{iid}\sim N(0,5^2)\), and for small error, we
  set \(\epsilon_{i,j}\overset{iid}\sim N(0,1^2)\).
\end{itemize}

All simulations were programmed in \texttt{R} 4.0.3 \citep{Rteam} and
the resulting data were analysed with \texttt{lme4} package (version
1.1-27.1) for fitting LME models \citep{Bates2015}. \texttt{HLMdiag}
(version 0.5.0) was used to extract residuals from the simulated models
\citep{loy2014hlmdiag}.

\subsection{Violating Distributional Assumptions}

Our design incorporates nine different misspecification scenarios and
one baseline scenario. The baseline's simulated data follows the above
data generation mechanism without any deliberate assumption violations.
Within each scenario, we have several settings based on all possible
combinations of group sizes, two variance component setups and possible
violations of model assumption(s) of interests. An example design matrix
for non-normality of residuals is shown in Table 1 below.

\vspace{-8pt}

\begin{table}[H]

\caption{\label{tab:example_design}Design Setting Example}
\centering
\fontsize{8}{10}\selectfont
\begin{tabular}[t]{lll}
\toprule
Variance Setting & Balance Setting & Misspecification\\
\midrule
\cellcolor{gray!6}{High Variance Error} & \cellcolor{gray!6}{Same-sized Groups} & \cellcolor{gray!6}{Non\_normality}\\
High Variance Random Effects & Same-sized Groups & Non\_normality\\
\cellcolor{gray!6}{High Variance Error} & \cellcolor{gray!6}{Balanced Groups} & \cellcolor{gray!6}{Non\_normality}\\
High Variance Random Effects & Balanced Groups & Non\_normality\\
\cellcolor{gray!6}{High Variance Error} & \cellcolor{gray!6}{Unbalanced Groups} & \cellcolor{gray!6}{Non\_normality}\\
\addlinespace
High Variance Random Effects & Unbalanced Groups & Non\_normality\\
\bottomrule
\end{tabular}
\end{table}

\vspace{-3pt}

There are 138 settings designed in total for this study, with 1000
artificial data sets and fitted models simulated for every setting. To
assess the performance of residual plots with model misspecifications,
we modified the data sets in each scenario (See the exact layout of the
entire design matrix in Table 3 of the Appendix).

\textbf{Scenario 1. Non-Normality}: The error term is drawn from a
skew-normal or a bimodal distribution using the \texttt{rpearson}
function from the \texttt{PearsonDS} R package (v1.2;
\citep{becker2021}) . Our study examines three levels of skewness of
errors in contrast with the normal distribution, which has a skewness
parameter of zero.

When errors are slightly skewed, the skewness parameter is 0.8 and
kurtosis of distribution is 4. For moderately skewed errors, the
skewness parameter is 1.5 and kurtosis is 6. For extreme skewness, the
skewness parameter is 3 and kurtosis is 11. In the bimodal setting, the
residual variance is drawn from a mixture of two normal distributions:
60\% from \(N(-1,1)\) and 40\% from \(N(1,1)\).

\textbf{Scenario 2. Non-constant Variance}: The error term is drawn from
distributions where the variance is a function of \(X_{i,j}\), as given
by:
\(\sigma_{Hetero}^2 = \sigma_{baseline}^2 + \lambda(X_{i,j}-\min(X_{i,j}))\)
where \(\lambda\), the heteroscedasticity factor, is set to 2, 4, or 8.
Note that \(\lambda\) = 0 yields the assumed normal distribution and a
larger heteroscedasticity factor of 8 ensures that the nonconstant
variance pattern of residuals is recognizably extreme
\citep{Schielzeth2020-gp}.

\textbf{Scenario 3. Non-linearity}: The data generation formula is
changed to
\(Y_{i,j}= \beta_0 + \beta_1 X_{i,j} + \beta_2 Z_{i,j}^2 + u_i + v_i X_i + \epsilon_{i,j}\).

\textbf{Scenario 4. Omitting a Fixed Effect}: We fit a reduced LME model
\(Z_{i,j}\):
\(Y_{i,j}= \beta_0 + \beta_1 X_{i,j} + u_i + v_i X_i + \epsilon_{i,j}\),
but the data are still generated from equation (3).

\textbf{Scenario 5. Non-normal Random Effects}: To violate the
multivariate normal assumption of random effects, we draw two random
effect components from skewed bivariate distributions where either the
random intercept or the random slope is skewed while keeping the other
random effect normally distributed. For simplicity, the non-normal
component follows the Pearson distribution similar to scenario 1 and is
always set as moderately skewed (skewness = 1.5, kurtosis = 6).

\textbf{Scenarios 6. 7. 8. (Combined Scenarios)}: Considering the
inherent complexity of hierarchical data in which more than one
assumptions on the residual quantities could be violated, we designed
scenarios in which homoscedasticity and normality of conditional
residuals, homoscedasticity of conditional residuals and linearity of
marginal residuals, or normality of conditional residuals and linearity
of marginal residuals are violated simultaneously. The process of
misspecification in these simulations follows the same procedures as in
Scenario 1 to 3.

\textbf{Scenario 9. Special Cases}: The intricacy of real-life data
means there are always other ways that the fitted models may fail to
accurately capture key characteristics existing in data. In this
scenario, we consider three types of model misspecification in addition
to typical assumptions of residuals.

• \textbf{\emph{Only random intercept}}: We also include settings where
there is only a random intercept component in the random effects. In
this case, we keep the random intercept component normally distributed.
When fitting the LME model, we misspecify the random effects structure
in still having both the random intercept and random slope.

• \textbf{\emph{Time variable}}: To simulate longitudinal data, we
construct one more fixed effect, \(T_{i,j}\) as a numerical sequence, to
mimic the time index variable for repeated observations taken within
each subject. When fitting the LME model, we omit this time index
variable.

• \textbf{\emph{Autocorrelated errors }}: Errors are drawn from a first
order autoregressive process to mimic the correlated errors that often
appear in longitudinal studies: \begin{equation} 
\epsilon_{t} =  \phi \epsilon_{t-1} + e_t\ \ \ \  \text{and} \ \ \ \ e_t \overset{iid}\sim N(0, \sigma_e) \end{equation}
where the autocorrelation coefficient \(\phi=0.4\) and the standard
deviation of error \(e_t = 1.5\).

Due to the technical difficulties in acquiring sufficient subjects'
responses and maintaining balanced group sizes in most longitudinal
studies, the simulated data sets in the latter two settings have three
types of dropout rates to account for these fluctuations: one with 50
groups and 25 observations per group to simulated large-scale, long-term
longitudinal studies with no dropout; one with 20 groups and from 8 to
10 observations per group to simulate relatively small and balanced
longitudinal studies with occasional dropouts; and the last one with 20
groups and 2 to 10 observations per group to simulate relatively small
and unbalanced longitudinal studies with multiple dropouts.

\subsection{Diagnosing Model Violations}

Our objective is to explore whether using visualizations of standardized
conditional residuals, Cholesky marginal residuals, and Mahalanobis
Distance would still be valid diagnostics tools when fitting LME models.
However, conducting visual diagnostics of more than
\(138 \times 1000 = 138,000\) plots is not reasonable for human
observers, so we used conventional diagnostics tests as feasible
substitutes for having to manually analyze residual plots. If the tests
perform as expected, then residual plots can be easily interpreted as in
the regression setting.

After purposefully violating model assumptions, we fit LME models using
REML.

\textbf{Normality test:} We adopt the Shapiro--Wilk (Shapiro) test for
testing normality of the error term \(\epsilon_{i,j}\) (conditional
residuals). In cases when there is only one random intercept in the data
sets but we misspecify the random effects structure as having both
random intercept and random slope components, we also use the Shapiro
test on the Mahalanobis Distance for normality of the random effects.

\textbf{Homoscedasticity test:} The Breusch--Pagan test (BP) is often
used to assess heteroscedasticity in the residuals of a linear
regression model \citep{breusch1979simple}. We use the BP test to check
homoscedasticity of the error terms. The null hypothesis is that the
variances of the error term do not depend on the independent variables
\(X_{i,j}\) and therefore are homoscedastic; if the test statistic has a
p-value below the significance level of 0.05, we reject the null
hypothesis and conclude that the error term is heteroscedastic.

\textbf{Assessing the distribution of the random effects:} Following the
recommendations from \citet{Singer2017-sd}, we use the
Kolmogorov-Smirnov (KS) test as a goodness-of-fit test on the
Mahalanobis QQ-plot to assess how well the distribution of the
Mahalanobis Distance agrees with the corresponding \(\chi^2_2\)
distribution in cases where there are random intercept and random slope,
\(\chi^2_1\) distribution in cases where there is only random intercept.

\textbf{Linearity test:} After extracting the Cholesky marginal
residuals, we fit linear and quadratic models to the marginal residuals
against the fitted. We then use Analysis of Variance (ANOVA)
\citep{bates1992statistical} to conduct a nested F-test on these models
to determine whether there is any discernible curvature in the error
terms.

\section{Evaluating Residual Plots}

Understanding the behaviors of residual plots is the ultimate goal for
our simulation study of LME model diagnostics. It is important to note
that different aspects of the specification are often entwined, and this
issue is particularly prominent in assessing the fit of LME models
\citep{Snijders2008-dm}. For example, it is possible that deviations
from random effects normality are caused by a misspecification of the
fixed effects structure \citep{mcculloch2011misspecifying} rather than
the residual distribution itself. We conduct all four diagnostics tests
in every scenario to provide a thorough examination of the residuals.

\textbf{Baseline setting}

We first explore the performance of residual diagnostics from properly
specified LME models. The results for these baseline models are shown in
Table 2. The Shapiro and ANOVA tests achieve the nominal type I error
rate in these settings despite changing variances and group sizes. The
baseline BP tests show slightly inflated type I error rate (9.5\% -
11.8\%). For normality of the estimated random effects, we also see
slighly higher rejection rates (7.1\% - 11.3\%) if the error terms are
more variable than the random effects, even when models are correctly
specified.

\vspace{-8pt}

\begin{table}[H]

\caption{\label{tab:good_result_table}Test Results from Residual Diagnostics of Properly Specified LME Models}
\centering
\fontsize{8}{10}\selectfont
\begin{tabular}[t]{llrrrr}
\toprule
Variance & Balance & Shapiro & BP.Test & KS.Test & ANOVA\\
\midrule
\cellcolor{gray!6}{High Variance Error} & \cellcolor{gray!6}{Same-sized Groups} & \cellcolor{gray!6}{0.045} & \cellcolor{gray!6}{0.100} & \cellcolor{gray!6}{0.104} & \cellcolor{gray!6}{0.046}\\
High Variance RE & Same-sized Groups & 0.049 & 0.101 & 0.004 & 0.059\\
\cellcolor{gray!6}{High Variance Error} & \cellcolor{gray!6}{Balanced Groups} & \cellcolor{gray!6}{0.053} & \cellcolor{gray!6}{0.098} & \cellcolor{gray!6}{0.085} & \cellcolor{gray!6}{0.050}\\
High Variance RE & Balanced Groups & 0.051 & 0.118 & 0.003 & 0.056\\
\cellcolor{gray!6}{High Variance Error} & \cellcolor{gray!6}{Unbalanced Groups} & \cellcolor{gray!6}{0.059} & \cellcolor{gray!6}{0.102} & \cellcolor{gray!6}{0.079} & \cellcolor{gray!6}{0.041}\\
\addlinespace
High Variance RE & Unbalanced Groups & 0.049 & 0.095 & 0.011 & 0.053\\
\bottomrule
\end{tabular}
\end{table}

\vspace{-7pt}

\textbf{Non-normal Error terms}

In scenario where we simulate non-normal errors, we would like the
Shapiro test to successfully pick up the non-normality of the errors
while the BP, KS and ANOVA tests maintain their nominal Type I error
rates, as this would allow straight-forward interpretation of the
residual plots.

• ~As seen in Figure 1-1, non-normality of the conditional residuals is
accurately detected in all simulations with skew-normal errors,
indicating that the skewness of conditional residuals can be discerned
from residual diagnostics. Interestingly, compared with the baseline
results, the non-normality of the estimated conditional residuals is not
apparent when errors have larger variances and follow bimodal
distributions. The connection between bimodality and variability of the
errors could be further explained by the phenomenon that when these two
distributions are more diffuse due to high variances, the resulting
mixture distribution would appear normal.

• ~More severe skewness in the conditional residuals leads to higher
false positive rates of heteroscedasticity, as shown in Figure 1-2,
especially if the error terms possess larger variances than the random
effects. Compared with the baseline, heteroscedasticity is not more
frequently detected among settings with errors in bimodal distribution
(baseline: 9.5\% - 11.8\%; bimodal: 7.6\% - 11.5\%), regardless of
variance components and group sizes.

• ~Using KS tests, we rarely detect false non-normality when the random
effects have lower variances in Figure 1-3 (0.2\% - 1.2\%), but this
error rates increases when the errors have higher variances (Type I
error rates increases to 7.1\% - 11.3\%).

• ~Meanwhile, ANOVA tests achieve nominal Type I error rates in Figure
1-4 (4.1\% - 5.9\%), meaning that no unusual quadratic structure is
detected and the Cholesky marginal residuals behave as we expected.

\begin{center}\includegraphics[width=1\linewidth]{Report_draft_files/figure-latex/normality-1} \end{center}

\vspace{-5pt}

\begin{spacing}{0.5}
\begingroup
\fontfamily{ppl}\fontsize{6.5}{16}\selectfont
Figure 1. The behaviors of residual quantities when the normality assumption of error terms was deliberately violated.
1-1 presents the rejection rate of Shapiro tests: Non-normality of conditional residuals is well detected, except bimodality with high variance errors. BP tests in 1-2 suggest that the increasing skewness of errors is linked to more severe false positive rates of heteroscedasticity. 1-3 and 1-4 show that normality of random effects and linearity of errors are not strongly deviated from the baseline statistics. 
\endgroup
\end{spacing}

~

\textbf{Non-Constant Variance}

In scenario where we simulate heteroscedastic errors, we would like the
BP test to successfully pick up the heteroscedasticity of the error
terms while the Shapiro, KS and ANOVA tests maintain their nominal Type
I error rates.

• ~When error terms are heteroscedastic and random effects are given
high variances, we observe higher rejection rates of Shapiro tests for
the conditional residuals in Figure 2-1. As the heteroscedasticity
factor gets higher, the false rejection rates increase. With more
unbalanced group sizes, we also see more frequent erroneous detection of
non-normality (15.6\% - 21\%) than with balanced groups (11.1\% -
17.4\%).

• ~BP tests on the conditional residuals (Figure 2-2) detect
heteroscedasticity more often with higher heteroscedasticity factors.
When the level of heteroscedasticity is relatively low (\(\lambda\) = 2
or 4), the heteroscedasticity of errors is detected more frequently if
the random effects have larger variances (20.4\% - 27.7\% with high
variance random effects and 11\% - 23.2\% with high variance errors). In
the extreme heteroscedasticity settings (\(\lambda\) = 8), the variances
of the error terms and the random effects don't appear to affect the
powers of BP tests.

• ~With error terms having higher variances, the false rejection rates
of KS tests in Figure 2-3 increase when \(\lambda\) becomes higher
(22.6\% - 26.7\% if \(\lambda=8\)), and remain low with high variance
random effects (0.3\% - 1.1\%). This suggests that analysts are likely
to identify non-normal random effects from looking at the Mahalanobis
distances when errors possess strong heteroscedasticity and large
variability.

• ~The ANOVA tests show that the estimated Cholesky marginal residuals
behave as expected, with the false rejection rate fluctuating around 5\%
in Figure 2-4.

\begin{center}\includegraphics[width=0.95\linewidth]{Report_draft_files/figure-latex/hetero-1} \end{center}

\vspace{-7pt}

\begin{spacing}{0.5}
\begingroup
\fontfamily{ppl}\fontsize{6}{16}\selectfont
Figure 2. The behaviors of residual quantities when homoscedasticity assumption of error terms was deliberately violated.
2-1 shows that conditional residuals are more likely to exhibit non-normality when error variances are low. BP tests in 2-2 suggest more severe heteroscedasticity with high variance random effects. The normality of random effects is more likely to break in 2-3 as errors possess higher variability. 2-4 shows that linearity of errors are not strongly deviated from the baseline. 
\endgroup
\end{spacing}

~

\begin{center}\includegraphics[width=1\linewidth]{Report_draft_files/figure-latex/fixed_effects-1} \end{center}

\vspace{-7pt}

\begin{spacing}{0.5}
\begingroup
\fontfamily{ppl}\fontsize{6.3}{16}\selectfont
Figure 3. The behaviors of residual quantities when the fixed effects were squared or omitted.
From 3-1 and 3-2, the misspecification of fixed effects would intensify non-normality and heteroscedasticity of low variance errors. The normality of random effects is more likely to break in 3-3 as errors possess higher variability and cluster sizes are unbalanced. 3-4 shows squaring one fixed effect leads to more quadratic distribution of errors. 
\endgroup
\end{spacing}

~

\textbf{Nonlinearity \& Omitted fixed effects}

In this scenario where we simulate one squared fixed effects term or
omit one fixed effects term, we would like the ANOVA test to pick up the
nonlinearity of the Cholesky residuals while the Shapiro test, BP test,
and KS test maintain their nominal Type I error rates.

• ~Fitting a model with only linear terms has great influence on the
conditional residual plots only when random effects are given higher
variability. In this case, there is also a high chance that the
estimated errors will not appear to be normally distributed (84.7\% -
86.3\%), and a moderate chance that they will appear to be homoscedastic
(34\% - 37.1\%). With errors possessing higher variances, the
significance tests no longer recognize the non-normality or
heteroscedasticity more often than in the baseline situation (Figure 3-1
\& 3-2).

• ~The spikes shown in Figure 3-4 for the ANOVA test rejection rates are
expected since one fixed effects term was squared. They are evidence
that misspecifying fixed effects have direct impacts on the behavior of
marginal residuals. This impacts, however, differ by the variability of
residual quantities: the rejection rates are between 97.9\% and 98.5\%
when errors have larger variances than the random effects and between
52.8\% and 57.1\% in the opposite situation. This again illustrates the
importance of taking into consideration of the ratio of residuals
variance when assessing model inadequacy by looking at residual plots.

• ~Omitting a fixed effect shows no strong repercussions on assessing
the distributional assumptions of residual quantities. Despite missing a
term, the fitted models generally yield the same rejection rates as in
the baseline situation. There are two exceptions as shown in Figure 3-1
and 3-3. There are high false positive rates of the conditional
residuals' normality assumption (32.3\% - 37.3\%) when we have high
variance random effects. The other is the elevated rejection rate of
random effects normality assumption (15.8\%) with high variance random
effects and unbalanced clusters.

~

\textbf{Non-normal Random Effects}

In the scenario where we simulate skewed bivariate normal random
effects, we would like the KS test to successfully pick up the
non-normality of the random effects while the Shapiro, BP and ANOVA
tests maintain their nominal Type I error rates. Overall, we discover
that using KS test on MD is not a reliable diagnostics tool for
misspecified LME:

• ~Regardless of whether the skewed component is the random intercept or
the random slope, the power of the KS test is higher when the errors
possess larger variances (8.7\% - 11.2\%) as opposed to cases when the
random effects are given higher variances (2.3\% - 3.6\%). Overall, the
highest power of KS tests is only 11.2\% across all settings, and in
cases when the random effects are given larger variances we see the
lowest detection power (well below 5\%). This implies that when errors
have higher variances, the random effects are more likely to appear
non-normal, no matter whether there is true model misspecification or
not, calling into question of the validity of using Mahalanobis QQ-plots
to diagnose random effects.

• ~The other test results, indicate that having skewed bivariate normal
random effects does not heavily impact the normal and constant-variance
properties of the estimated conditional residuals, nor the normality of
the Cholesky marginal residuals.

• ~We also simulate two-level data with only a random intercept
component while misspecifying the random effects structure as also
having a random slope component when fitting the LME model. Since there
is no underlying multivariate normal distribution but just the normally
distributed random intercept, we employ the Shapiro test on the random
effects to detect any potential non-normal behavior.

In this case, we see that normality assumption of error terms and the
Cholesky residuals generally holds well. For testing the homogeneity
assumption of the error terms, the Type I error rate of the BP test
fluctuates around 10\%.

To see if the residual diagnostics can pick up the fact that the random
effects do not follow multivariate normality distribution, we use the KS
test and observed that the power of the KS test vary between 33.6\% and
44.5\%. To see if we can diagnose that the random effects follows
univariate normality distribution, we use Shapiro test on the MD that
gives false rejection rates of 99.8\% to 100\% (Figure D-3). Overall, we
notice that using residual diagnostics to test the multivariate
normality assumption of the random effects are not reliable.

\begin{center}\includegraphics[width=1\linewidth]{Report_draft_files/figure-latex/RE-1} \end{center}

\vspace{-8pt}

\begin{spacing}{0.5}
\begingroup
\fontfamily{ppl}\fontsize{6.5}{16}\selectfont
Figure 4. The behaviors of residual quantities when the normality assumption of random effects was violated.
The influences of having skewed random effects distributions are genearly weak in inducing other assumption violations as shown in 4-1, 4-2 and 4-4. Higher chance of non-normal random effects continued to be associated with high variance errors in 4-3. 
\endgroup
\end{spacing}

~

\textbf{Heteroscedastic \& Moderately skewed error}

In the scenario where we simulate heteroscedastic errors with
\(\lambda\) = 2, 4, and 8 and also set the errors to be moderately
skewed, we would like the BP test to successfully pick up the
non-constant variance property and Shapiro test to pick up the
non-normality of the estimated conditional residuals while the KS and
ANOVA tests maintain their nominal Type I error rates.

• ~The non-normal behavior of the estimated conditional residuals would
be detected accurately using the Shapiro test regardless of the changing
variances and group sizes.

• ~The heteroscedastic pattern of the estimated conditional residuals
would be detected more accurately (50.3\% - 65.3\%) when errors have
larger variances, as against when the random effects have larger
variances (19.4\% - 31.6\%). As the heteroscedasticity factor,
\(\lambda\), gets larger, the power of BP tests does not necessarily get
stronger (Appendix Figure A-2), but rather stays in the range of 50\% -
65\% (high errors variance) and 25\% - 32\% (low errors variance).
Furthermore, holding \(\lambda\) constant, as cluster sizes of data
become more balanced, the estimated conditional residuals appear to be
less heteroscedastic.

• ~With heteroscedastic and skewed errors, MD tends to appear non-normal
when the errors have larger variances. As \(\lambda\) gets bigger, the
false rejection rate of KS test tends to get higher in cases when errors
have a larger variance (from 51.7\% - 80.2\%). Comparably, When the
errors have smaller variances, the MD seems to be behaving too
``normally'' with the rejection rate of the KS test being between 0.1\%
and 0.6\%, sometimes below the nominal Type I error rate.

• ~The ANOVA test results show that the estimated Cholesky marginal
residuals behave as we would expect, with the false rejection rate
fluctuating around 5\%.

~

\textbf{Heteroscedastic error \& Squaring fixed effects}

In the scenario where we simulate heteroscedastic errors with
\(\lambda\) = 2, 4, and 8 and squared one of the fixed effects, we would
like the BP test to successfully pick up the non-constant variance
property of the estimated conditional residuals and ANOVA test to pick
up the nonlinearity of the Cholesky residuals while the Shapiro and KS
tests maintain their nominal Type I error rates.

• ~When the errors have smaller variances, we could see that the false
rejection rates of the Shapiro test (16.9\% - 21.7\%) are much higher
than when the errors have comparably higher variances (4.8\% - 6.4\%).
(See Appendix Figure B-1)

• ~The heteroscedastic pattern of estimated conditional residuals are
detected most accurately when errors have smaller variances and
\(\lambda\) is at its lowest value. When \(\lambda\) increases to 4,
whether the errors or the random effects have larger variances, we see
that powers of BP tests are roughly the same (25\% - 30\%). When
\(\lambda\) = 8, the power increases especially in cases when errors
have higher variances. Overall, as \(\lambda\) increases from 2 to 8, we
see the overall powers of BP tests on estimated conditional residuals
will get higher. (See Appendix Figure B-2)

• ~The false rejection rates of KS tests are much higher (18.7\% -
50.4\%) when the errors have larger variances than when the errors had
lower variances (0.1\% - 1.1\%). (See Appendix Figure B-3)

• ~Checking the Cholesky marginal residuals using the ANOVA test, its
power decreases as \(\lambda\) gets higher. Regardless of whether the
cluster sizes are balanced or not, when the random effects have larger
variances, we see higher powers of ANOVA tests (44.3\% - 78.6\%); when
errors have higher variances, the powers of ANOVA tests are lower
(31.1\% - 45.7\%). (See Appendix Figure B-4)

~

\textbf{Squaring fixed effect \& Non-normal error}

In the scenario where we simulate non-normal error terms (moderate
skewed normal and bimodality) and induced nonlinearity, we would like
the Shapiro test to successfully pick up the non-normality of the
estimated conditional residuals and the ANOVA test to pick up the
nonlinearity of the Cholesky residuals while BP and KS tests maintain
their nominal Type I error rates.

• ~Overall, our Shapiro tests pick up the moderate skewness of the
errors better (77.2\% - 100\%) than the bimodal distribution (average
4.8\% - 12.3\%, one outlier rejection rate of 92.1\%). As the cluster
sizes get more unbalanced, the estimated conditional residuals appear
more normal (Appendix Figure C-1).

• ~When distribution of the error term is moderately skewed, the false
rejection rate of BP tests using conditional residuals is much higher
(23.6\% - 81.5\%) than when the errors are bimodal (8.4\% - 18.1\%).

• ~Overall, the false rejection rates of KS tests are not stable and are
likely to be higher (7.9\% - 71.5\%) when the errors have larger
variances compared than when the random effects have higher variances
(0.1\% - 0.9\%).

• ~Using the Cholesky marginal residuals, the ANOVA test has higher
power when the cluster sizes are more balanced, as well as in cases when
the error term is moderately skewed. Holding cluster sizes and skewness
type constant, the power of the ANOVA tests are much higher in cases
when the random effects have higher variances (26.5\% - 98.2\%) than
when errors have higher variances (7.9\% - 53.2\%). (Appendix Figure
C-4)

~

\textbf{Longitudinal settings}

In this scenario, we are interested in exploring the effects of model
structure misspecifications on residual plots. We considered omitting a
time index variable in longitudinal data set; and misspecifing AR(1)
errors as i.i.d errors. When fitting LME models, we use the standard
model mentioned in section 3.1 equation (3).

\textbf{\emph{Time variable case vs auto-correlated error}}

• ~The overall rejection rates of BP tests are in the range of 9.4\% to
15\%, with higher rejection rates when the longitudinal characteristics
appears in the auto-correlated error term (9.4\% - 14.6\%) compared to
the scenario when it appears in fixed-effect components as the time
index variable (8.5\% - 12.7\%).

\textbf{\emph{Shared results}}

• ~Holding all other factors constant, when random effects have larger
variances, we see that the false rejection rates of using BP tests on
testing the estimated conditional residuals' normality are slightly
higher than when errors have higher variances than random effects.

• ~Similar to previous scenarios, using KS tests on MD is not a stable
diagnostics tool (rejection rates range from 0.3\% to 69.5\%) (Appendix
Figure D-3); especially when the simulated data sets are designed to
have unbalanced clusters, we tend to see the non-normal behaviors of
random effects being exaggerated (rejection rate up to 69.5\%).

• ~With the estimated conditional residuals, Shapiro tests generally
achieve the nominal Type I error rate with occasionally higher rejection
rates when the data set has unbalanced group sizes and small sample
sizes (see small, unbalanced setting in Section 3). The ANOVA tests on
the Cholesky marginal residuals also maintain the nominal Type I error
rate.

\textbf{\emph{Special case}}

• ~When the random effects have larger variances, the group sizes are
balanced with a total of 50 groups in the data set, and the longitudinal
characteristics appears as the time index variable, we see one
``outlier'' Shapiro rejection rate of 100\% compared to the average 5\%
rejection rate in other longitudinal settings (Appendix Figure D-1).
After investigating the visuals of the standardized conditional
residuals, we determine that omitting the time index variable when we
have many groups and a large number of observations per group does skew
the distribution of the standardized conditional residuals to the extent
that using the Shapiro test is not advised.

\section{Discussion}

Our simulation study offers valuable insight into how residual
diagnostics can be used and interpreted for LME models.

\textbf{\emph{Single Misspecification}}: Our simulation results confirm
and strengthen the intertwined nature of residual behaviors: When only
one model assumption is misspecified, typically more than one model
assumption appears to be violated based on diagnostic plots. For
example, models with skewed normal error terms often appear to be
non-normal random effects and heteroscedastic errors. If the
homoscedasticity assumption is violated alone by having a high
heteroscedasticity factor, analysts will be likely to observe
non-normality of both conditional residuals and random effects along
with the expected non-constant variance pattern in residual plots.
Bimodality is a noteworthy exception: plots do not often show
non-normality of estimated errors or other inadequacy when true bimodal
errors have high variance.

The violation of the multivariate normal assumption of random effects
alone also does not result in strong deviations from any other model
assumptions. This finding in part is in agreement with
\citet{Schielzeth2020-gp} and supports the robustness of LME models in
the case of random effects misspecification.

Whether the longitudinal characteristic appears in the fixed effects as
a time index variable or in the errors as auto-correlated errors, the
distribution and variance patterns of the estimated standardized
conditional residuals and the Cholesky marginal residuals are not very
different from the correctly specified LME models. The one exception is
the MD, especially when the errors have larger variances, the MD tend to
show heavy deviations from the multivariate normality assumption. In
cases when we see errors with lower variance, the deviations are much
less discernible.

\textbf{\emph{Combined Misspecifications}}: When a pair of assumptions
are violated, analysts will tend to observe more issues in residual
plots. The two truly violated assumptions will be detected in their
corresponding diagnostic plots, along with some concerning issues with
other assumptions. For example, normality of random effects may be
falsely detected when non-normality and heteroscedasticity of errors
occur. The combination of non-linearity and non-constant variance
misspecifications in particular cause all four distributional
assumptions to be flagged.

\textbf{\emph{Residual Variability}}: The variability of errors and
random effects can lead to drastically different behaviors of residual
plots. We notice that normality of random effects are more likely to be
identified as problematic when error terms were given higher variance
than random effects. On the other hand, violating the linearity
assumption has more severe consequences if random effect variances are
larger. When there is a single misspecification, normality, linearity
and homoscedasticity assumptions of errors are more likely to be
violated in residual plots if their variances are smaller. The
interesting exception is heteroscedasticity induced by non-normality of
errors, which is more alarming if errors have larger variances. The
impacts of the relative magnitude of the residual variances are similar
in combined scenarios. Heteroscedasticity continues to be more
frequently detected where random effects are less variable than errors.
The normality assumption of errors, if not directly violated, tend to be
misdiagnosed more often with high variance random effects.

\textbf{\emph{Influences of Group Sizes}}: The distribution of cluster
sizes plays a more significant role in the combined scenarios.
Simulation results suggest that the normality and homoscedasticity
assumptions on the error terms are more likely to appear to be violated
with unbalanced clusters than evenly distributed ones. When linearity
and normality assumptions on the errors are violated, only unevenly
distributed clusters have their random effects flagged as non-normal,
and the heteroscedasticity and non-linearity of residuals become less
discernible as group sizes become more unbalanced.

Overall, our study revealed that employing a single diagnostic plot on
LME models' residuals is not sufficient to accurately assess the
validity of model assumptions. One remedy is to employ the lineup
protocol that will allows us to simultaneously assess several model
assumptions for a thorough investigation \citep{Loy2017-fo}. The
intertwined nature of these model specifications require practitioners
to examine not only a full set of diagnostic tools available for each
assumption, but also the ways in which the hierarchical structure of
fitted models was composed. Other important characteristics of the
interested data sets, for example cluster sizes, relative magnitude of
residual variances, omitted fixed effects and longitudinal sampling,
should also be checked to avoid exaggerating or overlooking any
violations to certain model assumptions. Failure to take these factors
into consideration will result in misinterpretation of diagnostics plots
and misjudgment of model assumptions, which can lead to biased model
estimates.

There are still questions left unaddressed beyond the scope of our
study. The validity of using least squares residuals for the diagnostic
purposes are still yet to be explored. Researchers could also look into
residual plots under more complex combinations of LME model
misspecifications, like three of more violations at the same time. In
comparison to the classic maximum likelihood approach there have also
been some alternative algorithms and methodologies developed for LME
model estimation, such as the Marquardt algorithm
\citep{proust2005estimation} and the robust estimation method
\citep{koller2016robustlmm}. We recommend more future research in the
direction of correctly evaluating expected diagnostic protocols for
these approaches.

\hypertarget{acknowledgement}{%
\section{Acknowledgement}\label{acknowledgement}}

We are deeply indebted to Professor Adam Loy from Carleton College who
kindly advised us throughout this research program.

\newpage

\section{Appendices}

\begin{table}

\caption{\label{tab:unnamed-chunk-1}Shapiro, BP, KS and ANOVA Test Results of all 138 Settings}
\centering
\resizebox{\linewidth}{!}{
\fontsize{2.27}{4.27}\selectfont
\begin{tabular}[t]{rllllrllllllrrrr}
\toprule
Setting & Variance & Balance & Normality & Linearity & H.factor & RE & FE & Hetero\_lin & Hetero\_norm & Lin\_norm & Special & Shapiro & BP.Test & KS.Test & ANOVA\\
\midrule
\cellcolor{gray!6}{1} & \cellcolor{gray!6}{High Variance Error} & \cellcolor{gray!6}{Same-sized Groups} & \cellcolor{gray!6}{norm} & \cellcolor{gray!6}{linear} & \cellcolor{gray!6}{0} & \cellcolor{gray!6}{norm\_re} & \cellcolor{gray!6}{full} & \cellcolor{gray!6}{linear\_homo} & \cellcolor{gray!6}{0\_skew} & \cellcolor{gray!6}{linear\_norm} & \cellcolor{gray!6}{standard} & \cellcolor{gray!6}{0.0450000} & \cellcolor{gray!6}{0.1000000} & \cellcolor{gray!6}{0.1040000} & \cellcolor{gray!6}{0.0460000}\\
2 & High Variance RE & Same-sized Groups & norm & linear & 0 & norm\_re & full & linear\_homo & 0\_skew & linear\_norm & standard & 0.0490000 & 0.1010000 & 0.0040000 & 0.0590000\\
\cellcolor{gray!6}{3} & \cellcolor{gray!6}{High Variance Error} & \cellcolor{gray!6}{Balanced Groups} & \cellcolor{gray!6}{norm} & \cellcolor{gray!6}{linear} & \cellcolor{gray!6}{0} & \cellcolor{gray!6}{norm\_re} & \cellcolor{gray!6}{full} & \cellcolor{gray!6}{linear\_homo} & \cellcolor{gray!6}{0\_skew} & \cellcolor{gray!6}{linear\_norm} & \cellcolor{gray!6}{standard} & \cellcolor{gray!6}{0.0530000} & \cellcolor{gray!6}{0.0980000} & \cellcolor{gray!6}{0.0850000} & \cellcolor{gray!6}{0.0500000}\\
4 & High Variance RE & Balanced Groups & norm & linear & 0 & norm\_re & full & linear\_homo & 0\_skew & linear\_norm & standard & 0.0510000 & 0.1180000 & 0.0030000 & 0.0560000\\
\cellcolor{gray!6}{5} & \cellcolor{gray!6}{High Variance Error} & \cellcolor{gray!6}{Unbalanced Groups} & \cellcolor{gray!6}{norm} & \cellcolor{gray!6}{linear} & \cellcolor{gray!6}{0} & \cellcolor{gray!6}{norm\_re} & \cellcolor{gray!6}{full} & \cellcolor{gray!6}{linear\_homo} & \cellcolor{gray!6}{0\_skew} & \cellcolor{gray!6}{linear\_norm} & \cellcolor{gray!6}{standard} & \cellcolor{gray!6}{0.0590000} & \cellcolor{gray!6}{0.1020000} & \cellcolor{gray!6}{0.0790000} & \cellcolor{gray!6}{0.0410000}\\
\addlinespace
6 & High Variance RE & Unbalanced Groups & norm & linear & 0 & norm\_re & full & linear\_homo & 0\_skew & linear\_norm & standard & 0.0490000 & 0.0950000 & 0.0110000 & 0.0530000\\
\cellcolor{gray!6}{7} & \cellcolor{gray!6}{High Variance Error} & \cellcolor{gray!6}{Same-sized Groups} & \cellcolor{gray!6}{skewness\_3} & \cellcolor{gray!6}{linear} & \cellcolor{gray!6}{0} & \cellcolor{gray!6}{norm\_re} & \cellcolor{gray!6}{full} & \cellcolor{gray!6}{linear\_homo} & \cellcolor{gray!6}{0\_skew} & \cellcolor{gray!6}{linear\_norm} & \cellcolor{gray!6}{standard} & \cellcolor{gray!6}{1.0000000} & \cellcolor{gray!6}{0.9880000} & \cellcolor{gray!6}{0.0880000} & \cellcolor{gray!6}{0.0420000}\\
8 & High Variance RE & Same-sized Groups & skewness\_3 & linear & 0 & norm\_re & full & linear\_homo & 0\_skew & linear\_norm & standard & 1.0000000 & 0.3760000 & 0.0120000 & 0.0430000\\
\cellcolor{gray!6}{9} & \cellcolor{gray!6}{High Variance Error} & \cellcolor{gray!6}{Balanced Groups} & \cellcolor{gray!6}{skewness\_3} & \cellcolor{gray!6}{linear} & \cellcolor{gray!6}{0} & \cellcolor{gray!6}{norm\_re} & \cellcolor{gray!6}{full} & \cellcolor{gray!6}{linear\_homo} & \cellcolor{gray!6}{0\_skew} & \cellcolor{gray!6}{linear\_norm} & \cellcolor{gray!6}{standard} & \cellcolor{gray!6}{1.0000000} & \cellcolor{gray!6}{0.9900000} & \cellcolor{gray!6}{0.1010000} & \cellcolor{gray!6}{0.0410000}\\
10 & High Variance RE & Balanced Groups & skewness\_3 & linear & 0 & norm\_re & full & linear\_homo & 0\_skew & linear\_norm & standard & 1.0000000 & 0.3920000 & 0.0060000 & 0.0430000\\
\addlinespace
\cellcolor{gray!6}{11} & \cellcolor{gray!6}{High Variance Error} & \cellcolor{gray!6}{Unbalanced Groups} & \cellcolor{gray!6}{skewness\_3} & \cellcolor{gray!6}{linear} & \cellcolor{gray!6}{0} & \cellcolor{gray!6}{norm\_re} & \cellcolor{gray!6}{full} & \cellcolor{gray!6}{linear\_homo} & \cellcolor{gray!6}{0\_skew} & \cellcolor{gray!6}{linear\_norm} & \cellcolor{gray!6}{standard} & \cellcolor{gray!6}{1.0000000} & \cellcolor{gray!6}{0.9760000} & \cellcolor{gray!6}{0.1000000} & \cellcolor{gray!6}{0.0430000}\\
12 & High Variance RE & Unbalanced Groups & skewness\_3 & linear & 0 & norm\_re & full & linear\_homo & 0\_skew & linear\_norm & standard & 1.0000000 & 0.4080000 & 0.0080000 & 0.0440000\\
\cellcolor{gray!6}{13} & \cellcolor{gray!6}{High Variance Error} & \cellcolor{gray!6}{Same-sized Groups} & \cellcolor{gray!6}{skewness\_1.5} & \cellcolor{gray!6}{linear} & \cellcolor{gray!6}{0} & \cellcolor{gray!6}{norm\_re} & \cellcolor{gray!6}{full} & \cellcolor{gray!6}{linear\_homo} & \cellcolor{gray!6}{0\_skew} & \cellcolor{gray!6}{linear\_norm} & \cellcolor{gray!6}{standard} & \cellcolor{gray!6}{1.0000000} & \cellcolor{gray!6}{0.8500000} & \cellcolor{gray!6}{0.0960000} & \cellcolor{gray!6}{0.0590000}\\
14 & High Variance RE & Same-sized Groups & skewness\_1.5 & linear & 0 & norm\_re & full & linear\_homo & 0\_skew & linear\_norm & standard & 1.0000000 & 0.2370000 & 0.0060000 & 0.0530000\\
\cellcolor{gray!6}{15} & \cellcolor{gray!6}{High Variance Error} & \cellcolor{gray!6}{Balanced Groups} & \cellcolor{gray!6}{skewness\_1.5} & \cellcolor{gray!6}{linear} & \cellcolor{gray!6}{0} & \cellcolor{gray!6}{norm\_re} & \cellcolor{gray!6}{full} & \cellcolor{gray!6}{linear\_homo} & \cellcolor{gray!6}{0\_skew} & \cellcolor{gray!6}{linear\_norm} & \cellcolor{gray!6}{standard} & \cellcolor{gray!6}{1.0000000} & \cellcolor{gray!6}{0.8670000} & \cellcolor{gray!6}{0.0830000} & \cellcolor{gray!6}{0.0480000}\\
\addlinespace
16 & High Variance RE & Balanced Groups & skewness\_1.5 & linear & 0 & norm\_re & full & linear\_homo & 0\_skew & linear\_norm & standard & 1.0000000 & 0.2490000 & 0.0040000 & 0.0460000\\
\cellcolor{gray!6}{17} & \cellcolor{gray!6}{High Variance Error} & \cellcolor{gray!6}{Unbalanced Groups} & \cellcolor{gray!6}{skewness\_1.5} & \cellcolor{gray!6}{linear} & \cellcolor{gray!6}{0} & \cellcolor{gray!6}{norm\_re} & \cellcolor{gray!6}{full} & \cellcolor{gray!6}{linear\_homo} & \cellcolor{gray!6}{0\_skew} & \cellcolor{gray!6}{linear\_norm} & \cellcolor{gray!6}{standard} & \cellcolor{gray!6}{1.0000000} & \cellcolor{gray!6}{0.8290000} & \cellcolor{gray!6}{0.0740000} & \cellcolor{gray!6}{0.0510000}\\
18 & High Variance RE & Unbalanced Groups & skewness\_1.5 & linear & 0 & norm\_re & full & linear\_homo & 0\_skew & linear\_norm & standard & 1.0000000 & 0.2460000 & 0.0040000 & 0.0500000\\
\cellcolor{gray!6}{19} & \cellcolor{gray!6}{High Variance Error} & \cellcolor{gray!6}{Same-sized Groups} & \cellcolor{gray!6}{skewness\_0.8} & \cellcolor{gray!6}{linear} & \cellcolor{gray!6}{0} & \cellcolor{gray!6}{norm\_re} & \cellcolor{gray!6}{full} & \cellcolor{gray!6}{linear\_homo} & \cellcolor{gray!6}{0\_skew} & \cellcolor{gray!6}{linear\_norm} & \cellcolor{gray!6}{standard} & \cellcolor{gray!6}{1.0000000} & \cellcolor{gray!6}{0.5000000} & \cellcolor{gray!6}{0.0920000} & \cellcolor{gray!6}{0.0610000}\\
20 & High Variance RE & Same-sized Groups & skewness\_0.8 & linear & 0 & norm\_re & full & linear\_homo & 0\_skew & linear\_norm & standard & 1.0000000 & 0.1490000 & 0.0060000 & 0.0410000\\
\addlinespace
\cellcolor{gray!6}{21} & \cellcolor{gray!6}{High Variance Error} & \cellcolor{gray!6}{Balanced Groups} & \cellcolor{gray!6}{skewness\_0.8} & \cellcolor{gray!6}{linear} & \cellcolor{gray!6}{0} & \cellcolor{gray!6}{norm\_re} & \cellcolor{gray!6}{full} & \cellcolor{gray!6}{linear\_homo} & \cellcolor{gray!6}{0\_skew} & \cellcolor{gray!6}{linear\_norm} & \cellcolor{gray!6}{standard} & \cellcolor{gray!6}{1.0000000} & \cellcolor{gray!6}{0.5130000} & \cellcolor{gray!6}{0.0910000} & \cellcolor{gray!6}{0.0530000}\\
22 & High Variance RE & Balanced Groups & skewness\_0.8 & linear & 0 & norm\_re & full & linear\_homo & 0\_skew & linear\_norm & standard & 1.0000000 & 0.1530000 & 0.0020000 & 0.0580000\\
\cellcolor{gray!6}{23} & \cellcolor{gray!6}{High Variance Error} & \cellcolor{gray!6}{Unbalanced Groups} & \cellcolor{gray!6}{skewness\_0.8} & \cellcolor{gray!6}{linear} & \cellcolor{gray!6}{0} & \cellcolor{gray!6}{norm\_re} & \cellcolor{gray!6}{full} & \cellcolor{gray!6}{linear\_homo} & \cellcolor{gray!6}{0\_skew} & \cellcolor{gray!6}{linear\_norm} & \cellcolor{gray!6}{standard} & \cellcolor{gray!6}{1.0000000} & \cellcolor{gray!6}{0.4640000} & \cellcolor{gray!6}{0.0710000} & \cellcolor{gray!6}{0.0590000}\\
24 & High Variance RE & Unbalanced Groups & skewness\_0.8 & linear & 0 & norm\_re & full & linear\_homo & 0\_skew & linear\_norm & standard & 1.0000000 & 0.1640000 & 0.0030000 & 0.0520000\\
\cellcolor{gray!6}{25} & \cellcolor{gray!6}{High Variance Error} & \cellcolor{gray!6}{Same-sized Groups} & \cellcolor{gray!6}{bimodal} & \cellcolor{gray!6}{linear} & \cellcolor{gray!6}{0} & \cellcolor{gray!6}{norm\_re} & \cellcolor{gray!6}{full} & \cellcolor{gray!6}{linear\_homo} & \cellcolor{gray!6}{0\_skew} & \cellcolor{gray!6}{linear\_norm} & \cellcolor{gray!6}{standard} & \cellcolor{gray!6}{0.0440000} & \cellcolor{gray!6}{0.1050000} & \cellcolor{gray!6}{0.0950000} & \cellcolor{gray!6}{0.0380000}\\
\addlinespace
26 & High Variance RE & Same-sized Groups & bimodal & linear & 0 & norm\_re & full & linear\_homo & 0\_skew & linear\_norm & standard & 0.9380000 & 0.0940000 & 0.0040000 & 0.0510000\\
\cellcolor{gray!6}{27} & \cellcolor{gray!6}{High Variance Error} & \cellcolor{gray!6}{Balanced Groups} & \cellcolor{gray!6}{bimodal} & \cellcolor{gray!6}{linear} & \cellcolor{gray!6}{0} & \cellcolor{gray!6}{norm\_re} & \cellcolor{gray!6}{full} & \cellcolor{gray!6}{linear\_homo} & \cellcolor{gray!6}{0\_skew} & \cellcolor{gray!6}{linear\_norm} & \cellcolor{gray!6}{standard} & \cellcolor{gray!6}{0.0490000} & \cellcolor{gray!6}{0.1150000} & \cellcolor{gray!6}{0.1130000} & \cellcolor{gray!6}{0.0440000}\\
28 & High Variance RE & Balanced Groups & bimodal & linear & 0 & norm\_re & full & linear\_homo & 0\_skew & linear\_norm & standard & 0.9420000 & 0.0780000 & 0.0090000 & 0.0410000\\
\cellcolor{gray!6}{29} & \cellcolor{gray!6}{High Variance Error} & \cellcolor{gray!6}{Unbalanced Groups} & \cellcolor{gray!6}{bimodal} & \cellcolor{gray!6}{linear} & \cellcolor{gray!6}{0} & \cellcolor{gray!6}{norm\_re} & \cellcolor{gray!6}{full} & \cellcolor{gray!6}{linear\_homo} & \cellcolor{gray!6}{0\_skew} & \cellcolor{gray!6}{linear\_norm} & \cellcolor{gray!6}{standard} & \cellcolor{gray!6}{0.0470000} & \cellcolor{gray!6}{0.0970000} & \cellcolor{gray!6}{0.0950000} & \cellcolor{gray!6}{0.0510000}\\
30 & High Variance RE & Unbalanced Groups & bimodal & linear & 0 & norm\_re & full & linear\_homo & 0\_skew & linear\_norm & standard & 0.9510000 & 0.0760000 & 0.0030000 & 0.0460000\\
\addlinespace
\cellcolor{gray!6}{31} & \cellcolor{gray!6}{High Variance Error} & \cellcolor{gray!6}{Same-sized Groups} & \cellcolor{gray!6}{norm} & \cellcolor{gray!6}{linear} & \cellcolor{gray!6}{2} & \cellcolor{gray!6}{norm\_re} & \cellcolor{gray!6}{full} & \cellcolor{gray!6}{linear\_homo} & \cellcolor{gray!6}{0\_skew} & \cellcolor{gray!6}{linear\_norm} & \cellcolor{gray!6}{standard} & \cellcolor{gray!6}{0.0550000} & \cellcolor{gray!6}{0.1170000} & \cellcolor{gray!6}{0.1510000} & \cellcolor{gray!6}{0.0530000}\\
32 & High Variance RE & Same-sized Groups & norm & linear & 2 & norm\_re & full & linear\_homo & 0\_skew & linear\_norm & standard & 0.1120000 & 0.2040000 & 0.0070000 & 0.0410000\\
\cellcolor{gray!6}{33} & \cellcolor{gray!6}{High Variance Error} & \cellcolor{gray!6}{Balanced Groups} & \cellcolor{gray!6}{norm} & \cellcolor{gray!6}{linear} & \cellcolor{gray!6}{2} & \cellcolor{gray!6}{norm\_re} & \cellcolor{gray!6}{full} & \cellcolor{gray!6}{linear\_homo} & \cellcolor{gray!6}{0\_skew} & \cellcolor{gray!6}{linear\_norm} & \cellcolor{gray!6}{standard} & \cellcolor{gray!6}{0.0510000} & \cellcolor{gray!6}{0.1100000} & \cellcolor{gray!6}{0.1360000} & \cellcolor{gray!6}{0.0500000}\\
34 & High Variance RE & Balanced Groups & norm & linear & 2 & norm\_re & full & linear\_homo & 0\_skew & linear\_norm & standard & 0.1110000 & 0.2040000 & 0.0100000 & 0.0410000\\
\cellcolor{gray!6}{35} & \cellcolor{gray!6}{High Variance Error} & \cellcolor{gray!6}{Unbalanced Groups} & \cellcolor{gray!6}{norm} & \cellcolor{gray!6}{linear} & \cellcolor{gray!6}{2} & \cellcolor{gray!6}{norm\_re} & \cellcolor{gray!6}{full} & \cellcolor{gray!6}{linear\_homo} & \cellcolor{gray!6}{0\_skew} & \cellcolor{gray!6}{linear\_norm} & \cellcolor{gray!6}{standard} & \cellcolor{gray!6}{0.0550000} & \cellcolor{gray!6}{0.1150000} & \cellcolor{gray!6}{0.1340000} & \cellcolor{gray!6}{0.0410000}\\
\addlinespace
36 & High Variance RE & Unbalanced Groups & norm & linear & 2 & norm\_re & full & linear\_homo & 0\_skew & linear\_norm & standard & 0.1560000 & 0.2180000 & 0.0050000 & 0.0520000\\
\cellcolor{gray!6}{37} & \cellcolor{gray!6}{High Variance Error} & \cellcolor{gray!6}{Same-sized Groups} & \cellcolor{gray!6}{norm} & \cellcolor{gray!6}{linear} & \cellcolor{gray!6}{4} & \cellcolor{gray!6}{norm\_re} & \cellcolor{gray!6}{full} & \cellcolor{gray!6}{linear\_homo} & \cellcolor{gray!6}{0\_skew} & \cellcolor{gray!6}{linear\_norm} & \cellcolor{gray!6}{standard} & \cellcolor{gray!6}{0.0540000} & \cellcolor{gray!6}{0.1500000} & \cellcolor{gray!6}{0.1680000} & \cellcolor{gray!6}{0.0420000}\\
38 & High Variance RE & Same-sized Groups & norm & linear & 4 & norm\_re & full & linear\_homo & 0\_skew & linear\_norm & standard & 0.1740000 & 0.2370000 & 0.0070000 & 0.0580000\\
\cellcolor{gray!6}{39} & \cellcolor{gray!6}{High Variance Error} & \cellcolor{gray!6}{Balanced Groups} & \cellcolor{gray!6}{norm} & \cellcolor{gray!6}{linear} & \cellcolor{gray!6}{4} & \cellcolor{gray!6}{norm\_re} & \cellcolor{gray!6}{full} & \cellcolor{gray!6}{linear\_homo} & \cellcolor{gray!6}{0\_skew} & \cellcolor{gray!6}{linear\_norm} & \cellcolor{gray!6}{standard} & \cellcolor{gray!6}{0.0540000} & \cellcolor{gray!6}{0.1620000} & \cellcolor{gray!6}{0.1850000} & \cellcolor{gray!6}{0.0390000}\\
40 & High Variance RE & Balanced Groups & norm & linear & 4 & norm\_re & full & linear\_homo & 0\_skew & linear\_norm & standard & 0.1660000 & 0.2440000 & 0.0040000 & 0.0430000\\
\addlinespace
\cellcolor{gray!6}{41} & \cellcolor{gray!6}{High Variance Error} & \cellcolor{gray!6}{Unbalanced Groups} & \cellcolor{gray!6}{norm} & \cellcolor{gray!6}{linear} & \cellcolor{gray!6}{4} & \cellcolor{gray!6}{norm\_re} & \cellcolor{gray!6}{full} & \cellcolor{gray!6}{linear\_homo} & \cellcolor{gray!6}{0\_skew} & \cellcolor{gray!6}{linear\_norm} & \cellcolor{gray!6}{standard} & \cellcolor{gray!6}{0.0440000} & \cellcolor{gray!6}{0.1570000} & \cellcolor{gray!6}{0.1420000} & \cellcolor{gray!6}{0.0450000}\\
42 & High Variance RE & Unbalanced Groups & norm & linear & 4 & norm\_re & full & linear\_homo & 0\_skew & linear\_norm & standard & 0.1950000 & 0.2770000 & 0.0060000 & 0.0420000\\
\cellcolor{gray!6}{43} & \cellcolor{gray!6}{High Variance Error} & \cellcolor{gray!6}{Same-sized Groups} & \cellcolor{gray!6}{norm} & \cellcolor{gray!6}{linear} & \cellcolor{gray!6}{8} & \cellcolor{gray!6}{norm\_re} & \cellcolor{gray!6}{full} & \cellcolor{gray!6}{linear\_homo} & \cellcolor{gray!6}{0\_skew} & \cellcolor{gray!6}{linear\_norm} & \cellcolor{gray!6}{standard} & \cellcolor{gray!6}{0.0570000} & \cellcolor{gray!6}{0.2320000} & \cellcolor{gray!6}{0.2670000} & \cellcolor{gray!6}{0.0420000}\\
44 & High Variance RE & Same-sized Groups & norm & linear & 8 & norm\_re & full & linear\_homo & 0\_skew & linear\_norm & standard & 0.1560000 & 0.2380000 & 0.0050000 & 0.0520000\\
\cellcolor{gray!6}{45} & \cellcolor{gray!6}{High Variance Error} & \cellcolor{gray!6}{Balanced Groups} & \cellcolor{gray!6}{norm} & \cellcolor{gray!6}{linear} & \cellcolor{gray!6}{8} & \cellcolor{gray!6}{norm\_re} & \cellcolor{gray!6}{full} & \cellcolor{gray!6}{linear\_homo} & \cellcolor{gray!6}{0\_skew} & \cellcolor{gray!6}{linear\_norm} & \cellcolor{gray!6}{standard} & \cellcolor{gray!6}{0.0510000} & \cellcolor{gray!6}{0.2220000} & \cellcolor{gray!6}{0.2330000} & \cellcolor{gray!6}{0.0540000}\\
\addlinespace
46 & High Variance RE & Balanced Groups & norm & linear & 8 & norm\_re & full & linear\_homo & 0\_skew & linear\_norm & standard & 0.1620000 & 0.2610000 & 0.0060000 & 0.0480000\\
\cellcolor{gray!6}{47} & \cellcolor{gray!6}{High Variance Error} & \cellcolor{gray!6}{Unbalanced Groups} & \cellcolor{gray!6}{norm} & \cellcolor{gray!6}{linear} & \cellcolor{gray!6}{8} & \cellcolor{gray!6}{norm\_re} & \cellcolor{gray!6}{full} & \cellcolor{gray!6}{linear\_homo} & \cellcolor{gray!6}{0\_skew} & \cellcolor{gray!6}{linear\_norm} & \cellcolor{gray!6}{standard} & \cellcolor{gray!6}{0.0500000} & \cellcolor{gray!6}{0.2230000} & \cellcolor{gray!6}{0.2260000} & \cellcolor{gray!6}{0.0340000}\\
48 & High Variance RE & Unbalanced Groups & norm & linear & 8 & norm\_re & full & linear\_homo & 0\_skew & linear\_norm & standard & 0.2100000 & 0.2360000 & 0.0030000 & 0.0500000\\
\cellcolor{gray!6}{49} & \cellcolor{gray!6}{High Variance Error} & \cellcolor{gray!6}{Same-sized Groups} & \cellcolor{gray!6}{norm} & \cellcolor{gray!6}{sq} & \cellcolor{gray!6}{0} & \cellcolor{gray!6}{norm\_re} & \cellcolor{gray!6}{full} & \cellcolor{gray!6}{linear\_homo} & \cellcolor{gray!6}{0\_skew} & \cellcolor{gray!6}{linear\_norm} & \cellcolor{gray!6}{standard} & \cellcolor{gray!6}{0.0490000} & \cellcolor{gray!6}{0.0890000} & \cellcolor{gray!6}{0.0990000} & \cellcolor{gray!6}{0.5440000}\\
50 & High Variance RE & Same-sized Groups & norm & sq & 0 & norm\_re & full & linear\_homo & 0\_skew & linear\_norm & standard & 0.8470000 & 0.3400000 & 0.0030000 & 0.9830000\\
\addlinespace
\cellcolor{gray!6}{51} & \cellcolor{gray!6}{High Variance Error} & \cellcolor{gray!6}{Balanced Groups} & \cellcolor{gray!6}{norm} & \cellcolor{gray!6}{sq} & \cellcolor{gray!6}{0} & \cellcolor{gray!6}{norm\_re} & \cellcolor{gray!6}{full} & \cellcolor{gray!6}{linear\_homo} & \cellcolor{gray!6}{0\_skew} & \cellcolor{gray!6}{linear\_norm} & \cellcolor{gray!6}{standard} & \cellcolor{gray!6}{0.0700000} & \cellcolor{gray!6}{0.1090000} & \cellcolor{gray!6}{0.0950000} & \cellcolor{gray!6}{0.5280000}\\
52 & High Variance RE & Balanced Groups & norm & sq & 0 & norm\_re & full & linear\_homo & 0\_skew & linear\_norm & standard & 0.8500000 & 0.3710000 & 0.0040000 & 0.9790000\\
\cellcolor{gray!6}{53} & \cellcolor{gray!6}{High Variance Error} & \cellcolor{gray!6}{Unbalanced Groups} & \cellcolor{gray!6}{norm} & \cellcolor{gray!6}{sq} & \cellcolor{gray!6}{0} & \cellcolor{gray!6}{norm\_re} & \cellcolor{gray!6}{full} & \cellcolor{gray!6}{linear\_homo} & \cellcolor{gray!6}{0\_skew} & \cellcolor{gray!6}{linear\_norm} & \cellcolor{gray!6}{standard} & \cellcolor{gray!6}{0.0610000} & \cellcolor{gray!6}{0.1030000} & \cellcolor{gray!6}{0.0760000} & \cellcolor{gray!6}{0.5710000}\\
54 & High Variance RE & Unbalanced Groups & norm & sq & 0 & norm\_re & full & linear\_homo & 0\_skew & linear\_norm & standard & 0.8630000 & 0.3690000 & 0.0090000 & 0.9860000\\
\cellcolor{gray!6}{55} & \cellcolor{gray!6}{High Variance Error} & \cellcolor{gray!6}{Same-sized Groups} & \cellcolor{gray!6}{norm} & \cellcolor{gray!6}{linear} & \cellcolor{gray!6}{0} & \cellcolor{gray!6}{mildly\_skewed\_re\_intercept} & \cellcolor{gray!6}{full} & \cellcolor{gray!6}{linear\_homo} & \cellcolor{gray!6}{0\_skew} & \cellcolor{gray!6}{linear\_norm} & \cellcolor{gray!6}{standard} & \cellcolor{gray!6}{0.0540000} & \cellcolor{gray!6}{0.1130000} & \cellcolor{gray!6}{0.0940000} & \cellcolor{gray!6}{0.0360000}\\
\addlinespace
56 & High Variance RE & Same-sized Groups & norm & linear & 0 & mildly\_skewed\_re\_intercept & full & linear\_homo & 0\_skew & linear\_norm & standard & 0.0440000 & 0.1000000 & 0.0280000 & 0.0520000\\
\cellcolor{gray!6}{57} & \cellcolor{gray!6}{High Variance Error} & \cellcolor{gray!6}{Balanced Groups} & \cellcolor{gray!6}{norm} & \cellcolor{gray!6}{linear} & \cellcolor{gray!6}{0} & \cellcolor{gray!6}{mildly\_skewed\_re\_intercept} & \cellcolor{gray!6}{full} & \cellcolor{gray!6}{linear\_homo} & \cellcolor{gray!6}{0\_skew} & \cellcolor{gray!6}{linear\_norm} & \cellcolor{gray!6}{standard} & \cellcolor{gray!6}{0.0490000} & \cellcolor{gray!6}{0.1100000} & \cellcolor{gray!6}{0.1120000} & \cellcolor{gray!6}{0.0610000}\\
58 & High Variance RE & Balanced Groups & norm & linear & 0 & mildly\_skewed\_re\_intercept & full & linear\_homo & 0\_skew & linear\_norm & standard & 0.0590000 & 0.0990000 & 0.0330000 & 0.0500000\\
\cellcolor{gray!6}{59} & \cellcolor{gray!6}{High Variance Error} & \cellcolor{gray!6}{Unbalanced Groups} & \cellcolor{gray!6}{norm} & \cellcolor{gray!6}{linear} & \cellcolor{gray!6}{0} & \cellcolor{gray!6}{mildly\_skewed\_re\_intercept} & \cellcolor{gray!6}{full} & \cellcolor{gray!6}{linear\_homo} & \cellcolor{gray!6}{0\_skew} & \cellcolor{gray!6}{linear\_norm} & \cellcolor{gray!6}{standard} & \cellcolor{gray!6}{0.0420000} & \cellcolor{gray!6}{0.0850000} & \cellcolor{gray!6}{0.1000000} & \cellcolor{gray!6}{0.0590000}\\
60 & High Variance RE & Unbalanced Groups & norm & linear & 0 & mildly\_skewed\_re\_intercept & full & linear\_homo & 0\_skew & linear\_norm & standard & 0.0560000 & 0.1200000 & 0.0360000 & 0.0420000\\
\addlinespace
\cellcolor{gray!6}{61} & \cellcolor{gray!6}{High Variance Error} & \cellcolor{gray!6}{Same-sized Groups} & \cellcolor{gray!6}{norm} & \cellcolor{gray!6}{linear} & \cellcolor{gray!6}{0} & \cellcolor{gray!6}{mildly\_skewed\_re\_slope} & \cellcolor{gray!6}{full} & \cellcolor{gray!6}{linear\_homo} & \cellcolor{gray!6}{0\_skew} & \cellcolor{gray!6}{linear\_norm} & \cellcolor{gray!6}{standard} & \cellcolor{gray!6}{0.0590000} & \cellcolor{gray!6}{0.0950000} & \cellcolor{gray!6}{0.1080000} & \cellcolor{gray!6}{0.0630000}\\
62 & High Variance RE & Same-sized Groups & norm & linear & 0 & mildly\_skewed\_re\_slope & full & linear\_homo & 0\_skew & linear\_norm & standard & 0.0640000 & 0.1190000 & 0.0310000 & 0.0570000\\
\cellcolor{gray!6}{63} & \cellcolor{gray!6}{High Variance Error} & \cellcolor{gray!6}{Balanced Groups} & \cellcolor{gray!6}{norm} & \cellcolor{gray!6}{linear} & \cellcolor{gray!6}{0} & \cellcolor{gray!6}{mildly\_skewed\_re\_slope} & \cellcolor{gray!6}{full} & \cellcolor{gray!6}{linear\_homo} & \cellcolor{gray!6}{0\_skew} & \cellcolor{gray!6}{linear\_norm} & \cellcolor{gray!6}{standard} & \cellcolor{gray!6}{0.0470000} & \cellcolor{gray!6}{0.1130000} & \cellcolor{gray!6}{0.0980000} & \cellcolor{gray!6}{0.0400000}\\
64 & High Variance RE & Balanced Groups & norm & linear & 0 & mildly\_skewed\_re\_slope & full & linear\_homo & 0\_skew & linear\_norm & standard & 0.0500000 & 0.0910000 & 0.0230000 & 0.0480000\\
\cellcolor{gray!6}{65} & \cellcolor{gray!6}{High Variance Error} & \cellcolor{gray!6}{Unbalanced Groups} & \cellcolor{gray!6}{norm} & \cellcolor{gray!6}{linear} & \cellcolor{gray!6}{0} & \cellcolor{gray!6}{mildly\_skewed\_re\_slope} & \cellcolor{gray!6}{full} & \cellcolor{gray!6}{linear\_homo} & \cellcolor{gray!6}{0\_skew} & \cellcolor{gray!6}{linear\_norm} & \cellcolor{gray!6}{standard} & \cellcolor{gray!6}{0.0460000} & \cellcolor{gray!6}{0.1030000} & \cellcolor{gray!6}{0.0870000} & \cellcolor{gray!6}{0.0520000}\\
\addlinespace
66 & High Variance RE & Unbalanced Groups & norm & linear & 0 & mildly\_skewed\_re\_slope & full & linear\_homo & 0\_skew & linear\_norm & standard & 0.0580000 & 0.1110000 & 0.0280000 & 0.0450000\\
\cellcolor{gray!6}{67} & \cellcolor{gray!6}{High Variance Error} & \cellcolor{gray!6}{Same-sized Groups} & \cellcolor{gray!6}{norm} & \cellcolor{gray!6}{linear} & \cellcolor{gray!6}{0} & \cellcolor{gray!6}{norm\_re} & \cellcolor{gray!6}{full} & \cellcolor{gray!6}{linear\_homo} & \cellcolor{gray!6}{2\_skew} & \cellcolor{gray!6}{linear\_norm} & \cellcolor{gray!6}{standard} & \cellcolor{gray!6}{1.0000000} & \cellcolor{gray!6}{0.6060000} & \cellcolor{gray!6}{0.6390000} & \cellcolor{gray!6}{0.0620000}\\
68 & High Variance RE & Same-sized Groups & norm & linear & 0 & norm\_re & full & linear\_homo & 2\_skew & linear\_norm & standard & 1.0000000 & 0.2150000 & 0.0060000 & 0.0540000\\
\cellcolor{gray!6}{69} & \cellcolor{gray!6}{High Variance Error} & \cellcolor{gray!6}{Balanced Groups} & \cellcolor{gray!6}{norm} & \cellcolor{gray!6}{linear} & \cellcolor{gray!6}{0} & \cellcolor{gray!6}{norm\_re} & \cellcolor{gray!6}{full} & \cellcolor{gray!6}{linear\_homo} & \cellcolor{gray!6}{2\_skew} & \cellcolor{gray!6}{linear\_norm} & \cellcolor{gray!6}{standard} & \cellcolor{gray!6}{1.0000000} & \cellcolor{gray!6}{0.5840000} & \cellcolor{gray!6}{0.5670000} & \cellcolor{gray!6}{0.0580000}\\
70 & High Variance RE & Balanced Groups & norm & linear & 0 & norm\_re & full & linear\_homo & 2\_skew & linear\_norm & standard & 1.0000000 & 0.1990000 & 0.0040000 & 0.0500000\\
\addlinespace
\cellcolor{gray!6}{71} & \cellcolor{gray!6}{High Variance Error} & \cellcolor{gray!6}{Unbalanced Groups} & \cellcolor{gray!6}{norm} & \cellcolor{gray!6}{linear} & \cellcolor{gray!6}{0} & \cellcolor{gray!6}{norm\_re} & \cellcolor{gray!6}{full} & \cellcolor{gray!6}{linear\_homo} & \cellcolor{gray!6}{2\_skew} & \cellcolor{gray!6}{linear\_norm} & \cellcolor{gray!6}{standard} & \cellcolor{gray!6}{1.0000000} & \cellcolor{gray!6}{0.5030000} & \cellcolor{gray!6}{0.5170000} & \cellcolor{gray!6}{0.0510000}\\
72 & High Variance RE & Unbalanced Groups & norm & linear & 0 & norm\_re & full & linear\_homo & 2\_skew & linear\_norm & standard & 1.0000000 & 0.1940000 & 0.0050000 & 0.0490000\\
\cellcolor{gray!6}{73} & \cellcolor{gray!6}{High Variance Error} & \cellcolor{gray!6}{Same-sized Groups} & \cellcolor{gray!6}{norm} & \cellcolor{gray!6}{linear} & \cellcolor{gray!6}{0} & \cellcolor{gray!6}{norm\_re} & \cellcolor{gray!6}{full} & \cellcolor{gray!6}{linear\_homo} & \cellcolor{gray!6}{4\_skew} & \cellcolor{gray!6}{linear\_norm} & \cellcolor{gray!6}{standard} & \cellcolor{gray!6}{1.0000000} & \cellcolor{gray!6}{0.6150000} & \cellcolor{gray!6}{0.6970000} & \cellcolor{gray!6}{0.0480000}\\
74 & High Variance RE & Same-sized Groups & norm & linear & 0 & norm\_re & full & linear\_homo & 4\_skew & linear\_norm & standard & 1.0000000 & 0.2720000 & 0.0040000 & 0.0520000\\
\cellcolor{gray!6}{75} & \cellcolor{gray!6}{High Variance Error} & \cellcolor{gray!6}{Balanced Groups} & \cellcolor{gray!6}{norm} & \cellcolor{gray!6}{linear} & \cellcolor{gray!6}{0} & \cellcolor{gray!6}{norm\_re} & \cellcolor{gray!6}{full} & \cellcolor{gray!6}{linear\_homo} & \cellcolor{gray!6}{4\_skew} & \cellcolor{gray!6}{linear\_norm} & \cellcolor{gray!6}{standard} & \cellcolor{gray!6}{1.0000000} & \cellcolor{gray!6}{0.6150000} & \cellcolor{gray!6}{0.5960000} & \cellcolor{gray!6}{0.0430000}\\
\addlinespace
76 & High Variance RE & Balanced Groups & norm & linear & 0 & norm\_re & full & linear\_homo & 4\_skew & linear\_norm & standard & 1.0000000 & 0.2500000 & 0.0040000 & 0.0460000\\
\cellcolor{gray!6}{77} & \cellcolor{gray!6}{High Variance Error} & \cellcolor{gray!6}{Unbalanced Groups} & \cellcolor{gray!6}{norm} & \cellcolor{gray!6}{linear} & \cellcolor{gray!6}{0} & \cellcolor{gray!6}{norm\_re} & \cellcolor{gray!6}{full} & \cellcolor{gray!6}{linear\_homo} & \cellcolor{gray!6}{4\_skew} & \cellcolor{gray!6}{linear\_norm} & \cellcolor{gray!6}{standard} & \cellcolor{gray!6}{1.0000000} & \cellcolor{gray!6}{0.5210000} & \cellcolor{gray!6}{0.6370000} & \cellcolor{gray!6}{0.0500000}\\
78 & High Variance RE & Unbalanced Groups & norm & linear & 0 & norm\_re & full & linear\_homo & 4\_skew & linear\_norm & standard & 1.0000000 & 0.2110000 & 0.0060000 & 0.0460000\\
\cellcolor{gray!6}{79} & \cellcolor{gray!6}{High Variance Error} & \cellcolor{gray!6}{Same-sized Groups} & \cellcolor{gray!6}{norm} & \cellcolor{gray!6}{linear} & \cellcolor{gray!6}{0} & \cellcolor{gray!6}{norm\_re} & \cellcolor{gray!6}{full} & \cellcolor{gray!6}{linear\_homo} & \cellcolor{gray!6}{8\_skew} & \cellcolor{gray!6}{linear\_norm} & \cellcolor{gray!6}{standard} & \cellcolor{gray!6}{1.0000000} & \cellcolor{gray!6}{0.6530000} & \cellcolor{gray!6}{0.7930000} & \cellcolor{gray!6}{0.0380000}\\
80 & High Variance RE & Same-sized Groups & norm & linear & 0 & norm\_re & full & linear\_homo & 8\_skew & linear\_norm & standard & 1.0000000 & 0.3160000 & 0.0010000 & 0.0520000\\
\addlinespace
\cellcolor{gray!6}{81} & \cellcolor{gray!6}{High Variance Error} & \cellcolor{gray!6}{Balanced Groups} & \cellcolor{gray!6}{norm} & \cellcolor{gray!6}{linear} & \cellcolor{gray!6}{0} & \cellcolor{gray!6}{norm\_re} & \cellcolor{gray!6}{full} & \cellcolor{gray!6}{linear\_homo} & \cellcolor{gray!6}{8\_skew} & \cellcolor{gray!6}{linear\_norm} & \cellcolor{gray!6}{standard} & \cellcolor{gray!6}{1.0000000} & \cellcolor{gray!6}{0.6300000} & \cellcolor{gray!6}{0.7440000} & \cellcolor{gray!6}{0.0470000}\\
82 & High Variance RE & Balanced Groups & norm & linear & 0 & norm\_re & full & linear\_homo & 8\_skew & linear\_norm & standard & 1.0000000 & 0.3030000 & 0.0030000 & 0.0560000\\
\cellcolor{gray!6}{83} & \cellcolor{gray!6}{High Variance Error} & \cellcolor{gray!6}{Unbalanced Groups} & \cellcolor{gray!6}{norm} & \cellcolor{gray!6}{linear} & \cellcolor{gray!6}{0} & \cellcolor{gray!6}{norm\_re} & \cellcolor{gray!6}{full} & \cellcolor{gray!6}{linear\_homo} & \cellcolor{gray!6}{8\_skew} & \cellcolor{gray!6}{linear\_norm} & \cellcolor{gray!6}{standard} & \cellcolor{gray!6}{1.0000000} & \cellcolor{gray!6}{0.5170000} & \cellcolor{gray!6}{0.8020000} & \cellcolor{gray!6}{0.0520000}\\
84 & High Variance RE & Unbalanced Groups & norm & linear & 0 & norm\_re & full & linear\_homo & 8\_skew & linear\_norm & standard & 1.0000000 & 0.2710000 & 0.0030000 & 0.0520000\\
\cellcolor{gray!6}{85} & \cellcolor{gray!6}{High Variance Error} & \cellcolor{gray!6}{Same-sized Groups} & \cellcolor{gray!6}{norm} & \cellcolor{gray!6}{linear} & \cellcolor{gray!6}{0} & \cellcolor{gray!6}{norm\_re} & \cellcolor{gray!6}{full} & \cellcolor{gray!6}{sq\_2} & \cellcolor{gray!6}{0\_skew} & \cellcolor{gray!6}{linear\_norm} & \cellcolor{gray!6}{standard} & \cellcolor{gray!6}{0.0490000} & \cellcolor{gray!6}{0.1780000} & \cellcolor{gray!6}{0.2170000} & \cellcolor{gray!6}{0.4470000}\\
\addlinespace
86 & High Variance RE & Same-sized Groups & norm & linear & 0 & norm\_re & full & sq\_2 & 0\_skew & linear\_norm & standard & 0.1740000 & 0.2320000 & 0.0020000 & 0.7860000\\
\cellcolor{gray!6}{87} & \cellcolor{gray!6}{High Variance Error} & \cellcolor{gray!6}{Balanced Groups} & \cellcolor{gray!6}{norm} & \cellcolor{gray!6}{linear} & \cellcolor{gray!6}{0} & \cellcolor{gray!6}{norm\_re} & \cellcolor{gray!6}{full} & \cellcolor{gray!6}{sq\_2} & \cellcolor{gray!6}{0\_skew} & \cellcolor{gray!6}{linear\_norm} & \cellcolor{gray!6}{standard} & \cellcolor{gray!6}{0.0480000} & \cellcolor{gray!6}{0.1570000} & \cellcolor{gray!6}{0.1870000} & \cellcolor{gray!6}{0.4390000}\\
88 & High Variance RE & Balanced Groups & norm & linear & 0 & norm\_re & full & sq\_2 & 0\_skew & linear\_norm & standard & 0.1710000 & 0.2580000 & 0.0030000 & 0.7800000\\
\cellcolor{gray!6}{89} & \cellcolor{gray!6}{High Variance Error} & \cellcolor{gray!6}{Unbalanced Groups} & \cellcolor{gray!6}{norm} & \cellcolor{gray!6}{linear} & \cellcolor{gray!6}{0} & \cellcolor{gray!6}{norm\_re} & \cellcolor{gray!6}{full} & \cellcolor{gray!6}{sq\_2} & \cellcolor{gray!6}{0\_skew} & \cellcolor{gray!6}{linear\_norm} & \cellcolor{gray!6}{standard} & \cellcolor{gray!6}{0.0500000} & \cellcolor{gray!6}{0.1810000} & \cellcolor{gray!6}{0.1680000} & \cellcolor{gray!6}{0.4570000}\\
90 & High Variance RE & Unbalanced Groups & norm & linear & 0 & norm\_re & full & sq\_2 & 0\_skew & linear\_norm & standard & 0.2140000 & 0.2910000 & 0.0080000 & 0.7850000\\
\addlinespace
\cellcolor{gray!6}{91} & \cellcolor{gray!6}{High Variance Error} & \cellcolor{gray!6}{Same-sized Groups} & \cellcolor{gray!6}{norm} & \cellcolor{gray!6}{linear} & \cellcolor{gray!6}{0} & \cellcolor{gray!6}{norm\_re} & \cellcolor{gray!6}{full} & \cellcolor{gray!6}{sq\_4} & \cellcolor{gray!6}{0\_skew} & \cellcolor{gray!6}{linear\_norm} & \cellcolor{gray!6}{standard} & \cellcolor{gray!6}{0.0540000} & \cellcolor{gray!6}{0.2920000} & \cellcolor{gray!6}{0.2950000} & \cellcolor{gray!6}{0.4020000}\\
92 & High Variance RE & Same-sized Groups & norm & linear & 0 & norm\_re & full & sq\_4 & 0\_skew & linear\_norm & standard & 0.2090000 & 0.2560000 & 0.0060000 & 0.6410000\\
\cellcolor{gray!6}{93} & \cellcolor{gray!6}{High Variance Error} & \cellcolor{gray!6}{Balanced Groups} & \cellcolor{gray!6}{norm} & \cellcolor{gray!6}{linear} & \cellcolor{gray!6}{0} & \cellcolor{gray!6}{norm\_re} & \cellcolor{gray!6}{full} & \cellcolor{gray!6}{sq\_4} & \cellcolor{gray!6}{0\_skew} & \cellcolor{gray!6}{linear\_norm} & \cellcolor{gray!6}{standard} & \cellcolor{gray!6}{0.0590000} & \cellcolor{gray!6}{0.2610000} & \cellcolor{gray!6}{0.2860000} & \cellcolor{gray!6}{0.3930000}\\
94 & High Variance RE & Balanced Groups & norm & linear & 0 & norm\_re & full & sq\_4 & 0\_skew & linear\_norm & standard & 0.1690000 & 0.2480000 & 0.0010000 & 0.6500000\\
\cellcolor{gray!6}{95} & \cellcolor{gray!6}{High Variance Error} & \cellcolor{gray!6}{Unbalanced Groups} & \cellcolor{gray!6}{norm} & \cellcolor{gray!6}{linear} & \cellcolor{gray!6}{0} & \cellcolor{gray!6}{norm\_re} & \cellcolor{gray!6}{full} & \cellcolor{gray!6}{sq\_4} & \cellcolor{gray!6}{0\_skew} & \cellcolor{gray!6}{linear\_norm} & \cellcolor{gray!6}{standard} & \cellcolor{gray!6}{0.0530000} & \cellcolor{gray!6}{0.3120000} & \cellcolor{gray!6}{0.2470000} & \cellcolor{gray!6}{0.4090000}\\
\addlinespace
96 & High Variance RE & Unbalanced Groups & norm & linear & 0 & norm\_re & full & sq\_4 & 0\_skew & linear\_norm & standard & 0.2170000 & 0.2640000 & 0.0110000 & 0.6410000\\
\cellcolor{gray!6}{97} & \cellcolor{gray!6}{High Variance Error} & \cellcolor{gray!6}{Same-sized Groups} & \cellcolor{gray!6}{norm} & \cellcolor{gray!6}{linear} & \cellcolor{gray!6}{0} & \cellcolor{gray!6}{norm\_re} & \cellcolor{gray!6}{full} & \cellcolor{gray!6}{sq\_8} & \cellcolor{gray!6}{0\_skew} & \cellcolor{gray!6}{linear\_norm} & \cellcolor{gray!6}{standard} & \cellcolor{gray!6}{0.0580000} & \cellcolor{gray!6}{0.4420000} & \cellcolor{gray!6}{0.5040000} & \cellcolor{gray!6}{0.3110000}\\
98 & High Variance RE & Same-sized Groups & norm & linear & 0 & norm\_re & full & sq\_8 & 0\_skew & linear\_norm & standard & 0.1900000 & 0.2530000 & 0.0040000 & 0.4500000\\
\cellcolor{gray!6}{99} & \cellcolor{gray!6}{High Variance Error} & \cellcolor{gray!6}{Balanced Groups} & \cellcolor{gray!6}{norm} & \cellcolor{gray!6}{linear} & \cellcolor{gray!6}{0} & \cellcolor{gray!6}{norm\_re} & \cellcolor{gray!6}{full} & \cellcolor{gray!6}{sq\_8} & \cellcolor{gray!6}{0\_skew} & \cellcolor{gray!6}{linear\_norm} & \cellcolor{gray!6}{standard} & \cellcolor{gray!6}{0.0550000} & \cellcolor{gray!6}{0.4800000} & \cellcolor{gray!6}{0.4880000} & \cellcolor{gray!6}{0.3190000}\\
100 & High Variance RE & Balanced Groups & norm & linear & 0 & norm\_re & full & sq\_8 & 0\_skew & linear\_norm & standard & 0.2150000 & 0.2670000 & 0.0030000 & 0.4430000\\
\addlinespace
\cellcolor{gray!6}{101} & \cellcolor{gray!6}{High Variance Error} & \cellcolor{gray!6}{Unbalanced Groups} & \cellcolor{gray!6}{norm} & \cellcolor{gray!6}{linear} & \cellcolor{gray!6}{0} & \cellcolor{gray!6}{norm\_re} & \cellcolor{gray!6}{full} & \cellcolor{gray!6}{sq\_8} & \cellcolor{gray!6}{0\_skew} & \cellcolor{gray!6}{linear\_norm} & \cellcolor{gray!6}{standard} & \cellcolor{gray!6}{0.0640000} & \cellcolor{gray!6}{0.5000000} & \cellcolor{gray!6}{0.4630000} & \cellcolor{gray!6}{0.3220000}\\
102 & High Variance RE & Unbalanced Groups & norm & linear & 0 & norm\_re & full & sq\_8 & 0\_skew & linear\_norm & standard & 0.2290000 & 0.2940000 & 0.0030000 & 0.4710000\\
\cellcolor{gray!6}{103} & \cellcolor{gray!6}{High Variance Error} & \cellcolor{gray!6}{Same-sized Groups} & \cellcolor{gray!6}{norm} & \cellcolor{gray!6}{linear} & \cellcolor{gray!6}{0} & \cellcolor{gray!6}{norm\_re} & \cellcolor{gray!6}{reduced} & \cellcolor{gray!6}{linear\_homo} & \cellcolor{gray!6}{0\_skew} & \cellcolor{gray!6}{linear\_norm} & \cellcolor{gray!6}{standard} & \cellcolor{gray!6}{0.0445860} & \cellcolor{gray!6}{0.1019108} & \cellcolor{gray!6}{0.0700637} & \cellcolor{gray!6}{0.0445860}\\
104 & High Variance RE & Same-sized Groups & norm & linear & 0 & norm\_re & reduced & linear\_homo & 0\_skew & linear\_norm & standard & 0.3734177 & 0.1012658 & 0.0000000 & 0.0445860\\
\cellcolor{gray!6}{105} & \cellcolor{gray!6}{High Variance Error} & \cellcolor{gray!6}{Balanced Groups} & \cellcolor{gray!6}{norm} & \cellcolor{gray!6}{linear} & \cellcolor{gray!6}{0} & \cellcolor{gray!6}{norm\_re} & \cellcolor{gray!6}{reduced} & \cellcolor{gray!6}{linear\_homo} & \cellcolor{gray!6}{0\_skew} & \cellcolor{gray!6}{linear\_norm} & \cellcolor{gray!6}{standard} & \cellcolor{gray!6}{0.0691824} & \cellcolor{gray!6}{0.0880503} & \cellcolor{gray!6}{0.0880503} & \cellcolor{gray!6}{0.0503145}\\
\addlinespace
106 & High Variance RE & Balanced Groups & norm & linear & 0 & norm\_re & reduced & linear\_homo & 0\_skew & linear\_norm & standard & 0.3375796 & 0.1019108 & 0.0000000 & 0.0445860\\
\cellcolor{gray!6}{107} & \cellcolor{gray!6}{High Variance Error} & \cellcolor{gray!6}{Unbalanced Groups} & \cellcolor{gray!6}{norm} & \cellcolor{gray!6}{linear} & \cellcolor{gray!6}{0} & \cellcolor{gray!6}{norm\_re} & \cellcolor{gray!6}{reduced} & \cellcolor{gray!6}{linear\_homo} & \cellcolor{gray!6}{0\_skew} & \cellcolor{gray!6}{linear\_norm} & \cellcolor{gray!6}{standard} & \cellcolor{gray!6}{0.0443038} & \cellcolor{gray!6}{0.1139240} & \cellcolor{gray!6}{0.1518987} & \cellcolor{gray!6}{0.0253165}\\
108 & High Variance RE & Unbalanced Groups & norm & linear & 0 & norm\_re & reduced & linear\_homo & 0\_skew & linear\_norm & standard & 0.3227848 & 0.1075949 & 0.0063291 & 0.0379747\\
\cellcolor{gray!6}{109} & \cellcolor{gray!6}{High Variance Error} & \cellcolor{gray!6}{Same-sized Groups} & \cellcolor{gray!6}{norm} & \cellcolor{gray!6}{linear} & \cellcolor{gray!6}{0} & \cellcolor{gray!6}{norm\_re} & \cellcolor{gray!6}{full} & \cellcolor{gray!6}{linear\_homo} & \cellcolor{gray!6}{0\_skew} & \cellcolor{gray!6}{reduced\_skew} & \cellcolor{gray!6}{standard} & \cellcolor{gray!6}{1.0000000} & \cellcolor{gray!6}{0.8150000} & \cellcolor{gray!6}{0.0740000} & \cellcolor{gray!6}{0.5320000}\\
110 & High Variance RE & Same-sized Groups & norm & linear & 0 & norm\_re & full & linear\_homo & 0\_skew & reduced\_skew & standard & 1.0000000 & 0.5030000 & 0.0070000 & 0.9820000\\
\addlinespace
\cellcolor{gray!6}{111} & \cellcolor{gray!6}{High Variance Error} & \cellcolor{gray!6}{Balanced Groups} & \cellcolor{gray!6}{norm} & \cellcolor{gray!6}{linear} & \cellcolor{gray!6}{0} & \cellcolor{gray!6}{norm\_re} & \cellcolor{gray!6}{full} & \cellcolor{gray!6}{linear\_homo} & \cellcolor{gray!6}{0\_skew} & \cellcolor{gray!6}{reduced\_skew} & \cellcolor{gray!6}{standard} & \cellcolor{gray!6}{1.0000000} & \cellcolor{gray!6}{0.4990000} & \cellcolor{gray!6}{0.5950000} & \cellcolor{gray!6}{0.1110000}\\
112 & High Variance RE & Balanced Groups & norm & linear & 0 & norm\_re & full & linear\_homo & 0\_skew & reduced\_skew & standard & 0.9740000 & 0.2420000 & 0.0040000 & 0.5810000\\
\cellcolor{gray!6}{113} & \cellcolor{gray!6}{High Variance Error} & \cellcolor{gray!6}{Unbalanced Groups} & \cellcolor{gray!6}{norm} & \cellcolor{gray!6}{linear} & \cellcolor{gray!6}{0} & \cellcolor{gray!6}{norm\_re} & \cellcolor{gray!6}{full} & \cellcolor{gray!6}{linear\_homo} & \cellcolor{gray!6}{0\_skew} & \cellcolor{gray!6}{reduced\_skew} & \cellcolor{gray!6}{standard} & \cellcolor{gray!6}{0.9900000} & \cellcolor{gray!6}{0.4900000} & \cellcolor{gray!6}{0.7150000} & \cellcolor{gray!6}{0.0950000}\\
114 & High Variance RE & Unbalanced Groups & norm & linear & 0 & norm\_re & full & linear\_homo & 0\_skew & reduced\_skew & standard & 0.7720000 & 0.2360000 & 0.0090000 & 0.3510000\\
\cellcolor{gray!6}{115} & \cellcolor{gray!6}{High Variance Error} & \cellcolor{gray!6}{Same-sized Groups} & \cellcolor{gray!6}{norm} & \cellcolor{gray!6}{linear} & \cellcolor{gray!6}{0} & \cellcolor{gray!6}{norm\_re} & \cellcolor{gray!6}{full} & \cellcolor{gray!6}{linear\_homo} & \cellcolor{gray!6}{0\_skew} & \cellcolor{gray!6}{reduced\_bimodal} & \cellcolor{gray!6}{standard} & \cellcolor{gray!6}{0.0480000} & \cellcolor{gray!6}{0.1020000} & \cellcolor{gray!6}{0.0790000} & \cellcolor{gray!6}{0.5030000}\\
\addlinespace
116 & High Variance RE & Same-sized Groups & norm & linear & 0 & norm\_re & full & linear\_homo & 0\_skew & reduced\_bimodal & standard & 0.9210000 & 0.1810000 & 0.0030000 & 0.9540000\\
\cellcolor{gray!6}{117} & \cellcolor{gray!6}{High Variance Error} & \cellcolor{gray!6}{Balanced Groups} & \cellcolor{gray!6}{norm} & \cellcolor{gray!6}{linear} & \cellcolor{gray!6}{0} & \cellcolor{gray!6}{norm\_re} & \cellcolor{gray!6}{full} & \cellcolor{gray!6}{linear\_homo} & \cellcolor{gray!6}{0\_skew} & \cellcolor{gray!6}{reduced\_bimodal} & \cellcolor{gray!6}{standard} & \cellcolor{gray!6}{0.0470000} & \cellcolor{gray!6}{0.1030000} & \cellcolor{gray!6}{0.6030000} & \cellcolor{gray!6}{0.1110000}\\
118 & High Variance RE & Balanced Groups & norm & linear & 0 & norm\_re & full & linear\_homo & 0\_skew & reduced\_bimodal & standard & 0.1230000 & 0.1250000 & 0.0010000 & 0.4140000\\
\cellcolor{gray!6}{119} & \cellcolor{gray!6}{High Variance Error} & \cellcolor{gray!6}{Unbalanced Groups} & \cellcolor{gray!6}{norm} & \cellcolor{gray!6}{linear} & \cellcolor{gray!6}{0} & \cellcolor{gray!6}{norm\_re} & \cellcolor{gray!6}{full} & \cellcolor{gray!6}{linear\_homo} & \cellcolor{gray!6}{0\_skew} & \cellcolor{gray!6}{reduced\_bimodal} & \cellcolor{gray!6}{standard} & \cellcolor{gray!6}{0.0520000} & \cellcolor{gray!6}{0.0840000} & \cellcolor{gray!6}{0.6760000} & \cellcolor{gray!6}{0.0790000}\\
120 & High Variance RE & Unbalanced Groups & norm & linear & 0 & norm\_re & full & linear\_homo & 0\_skew & reduced\_bimodal & standard & 0.0860000 & 0.1140000 & 0.0080000 & 0.2650000\\
\addlinespace
\cellcolor{gray!6}{121} & \cellcolor{gray!6}{High Variance Error} & \cellcolor{gray!6}{Same-sized Groups} & \cellcolor{gray!6}{norm} & \cellcolor{gray!6}{linear} & \cellcolor{gray!6}{0} & \cellcolor{gray!6}{norm\_re} & \cellcolor{gray!6}{full} & \cellcolor{gray!6}{linear\_homo} & \cellcolor{gray!6}{0\_skew} & \cellcolor{gray!6}{linear\_norm} & \cellcolor{gray!6}{ar\_error} & \cellcolor{gray!6}{0.0700000} & \cellcolor{gray!6}{0.1210000} & \cellcolor{gray!6}{0.0160000} & \cellcolor{gray!6}{0.0460000}\\
122 & High Variance RE & Same-sized Groups & norm & linear & 0 & norm\_re & full & linear\_homo & 0\_skew & linear\_norm & ar\_error & 0.0560000 & 0.1060000 & 0.0070000 & 0.0420000\\
\cellcolor{gray!6}{123} & \cellcolor{gray!6}{High Variance Error} & \cellcolor{gray!6}{Balanced Groups} & \cellcolor{gray!6}{norm} & \cellcolor{gray!6}{linear} & \cellcolor{gray!6}{0} & \cellcolor{gray!6}{norm\_re} & \cellcolor{gray!6}{full} & \cellcolor{gray!6}{linear\_homo} & \cellcolor{gray!6}{0\_skew} & \cellcolor{gray!6}{linear\_norm} & \cellcolor{gray!6}{ar\_error} & \cellcolor{gray!6}{0.0500000} & \cellcolor{gray!6}{0.0940000} & \cellcolor{gray!6}{0.3130000} & \cellcolor{gray!6}{0.0420000}\\
124 & High Variance RE & Balanced Groups & norm & linear & 0 & norm\_re & full & linear\_homo & 0\_skew & linear\_norm & ar\_error & 0.0550000 & 0.1260000 & 0.0030000 & 0.0400000\\
\cellcolor{gray!6}{125} & \cellcolor{gray!6}{High Variance Error} & \cellcolor{gray!6}{Unbalanced Groups} & \cellcolor{gray!6}{norm} & \cellcolor{gray!6}{linear} & \cellcolor{gray!6}{0} & \cellcolor{gray!6}{norm\_re} & \cellcolor{gray!6}{full} & \cellcolor{gray!6}{linear\_homo} & \cellcolor{gray!6}{0\_skew} & \cellcolor{gray!6}{linear\_norm} & \cellcolor{gray!6}{ar\_error} & \cellcolor{gray!6}{0.0360000} & \cellcolor{gray!6}{0.1270000} & \cellcolor{gray!6}{0.4480000} & \cellcolor{gray!6}{0.0460000}\\
\addlinespace
126 & High Variance RE & Unbalanced Groups & norm & linear & 0 & norm\_re & full & linear\_homo & 0\_skew & linear\_norm & ar\_error & 0.1360000 & 0.1460000 & 0.0100000 & 0.0360000\\
\cellcolor{gray!6}{127} & \cellcolor{gray!6}{High Variance Error} & \cellcolor{gray!6}{Same-sized Groups} & \cellcolor{gray!6}{norm} & \cellcolor{gray!6}{linear} & \cellcolor{gray!6}{0} & \cellcolor{gray!6}{norm\_re} & \cellcolor{gray!6}{full} & \cellcolor{gray!6}{linear\_homo} & \cellcolor{gray!6}{0\_skew} & \cellcolor{gray!6}{linear\_norm} & \cellcolor{gray!6}{re\_int} & \cellcolor{gray!6}{0.0440000} & \cellcolor{gray!6}{0.0920000} & \cellcolor{gray!6}{0.9980000} & \cellcolor{gray!6}{0.0420000}\\
128 & High Variance RE & Same-sized Groups & norm & linear & 0 & norm\_re & full & linear\_homo & 0\_skew & linear\_norm & re\_int & 0.0450000 & 0.1150000 & 1.0000000 & 0.0490000\\
\cellcolor{gray!6}{129} & \cellcolor{gray!6}{High Variance Error} & \cellcolor{gray!6}{Balanced Groups} & \cellcolor{gray!6}{norm} & \cellcolor{gray!6}{linear} & \cellcolor{gray!6}{0} & \cellcolor{gray!6}{norm\_re} & \cellcolor{gray!6}{full} & \cellcolor{gray!6}{linear\_homo} & \cellcolor{gray!6}{0\_skew} & \cellcolor{gray!6}{linear\_norm} & \cellcolor{gray!6}{re\_int} & \cellcolor{gray!6}{0.0500000} & \cellcolor{gray!6}{0.1090000} & \cellcolor{gray!6}{0.9990000} & \cellcolor{gray!6}{0.0470000}\\
130 & High Variance RE & Balanced Groups & norm & linear & 0 & norm\_re & full & linear\_homo & 0\_skew & linear\_norm & re\_int & 0.0460000 & 0.1140000 & 1.0000000 & 0.0440000\\
\addlinespace
\cellcolor{gray!6}{131} & \cellcolor{gray!6}{High Variance Error} & \cellcolor{gray!6}{Unbalanced Groups} & \cellcolor{gray!6}{norm} & \cellcolor{gray!6}{linear} & \cellcolor{gray!6}{0} & \cellcolor{gray!6}{norm\_re} & \cellcolor{gray!6}{full} & \cellcolor{gray!6}{linear\_homo} & \cellcolor{gray!6}{0\_skew} & \cellcolor{gray!6}{linear\_norm} & \cellcolor{gray!6}{re\_int} & \cellcolor{gray!6}{0.0480000} & \cellcolor{gray!6}{0.0920000} & \cellcolor{gray!6}{1.0000000} & \cellcolor{gray!6}{0.0470000}\\
132 & High Variance RE & Unbalanced Groups & norm & linear & 0 & norm\_re & full & linear\_homo & 0\_skew & linear\_norm & re\_int & 0.0540000 & 0.0930000 & 0.9990000 & 0.0490000\\
\cellcolor{gray!6}{133} & \cellcolor{gray!6}{High Variance Error} & \cellcolor{gray!6}{Same-sized Groups} & \cellcolor{gray!6}{norm} & \cellcolor{gray!6}{linear} & \cellcolor{gray!6}{0} & \cellcolor{gray!6}{norm\_re} & \cellcolor{gray!6}{full} & \cellcolor{gray!6}{linear\_homo} & \cellcolor{gray!6}{0\_skew} & \cellcolor{gray!6}{linear\_norm} & \cellcolor{gray!6}{time\_seq} & \cellcolor{gray!6}{0.0800000} & \cellcolor{gray!6}{0.0930000} & \cellcolor{gray!6}{0.5530000} & \cellcolor{gray!6}{0.0470000}\\
134 & High Variance RE & Same-sized Groups & norm & linear & 0 & norm\_re & full & linear\_homo & 0\_skew & linear\_norm & time\_seq & 1.0000000 & 0.0890000 & 0.0070000 & 0.0910000\\
\cellcolor{gray!6}{135} & \cellcolor{gray!6}{High Variance Error} & \cellcolor{gray!6}{Balanced Groups} & \cellcolor{gray!6}{norm} & \cellcolor{gray!6}{linear} & \cellcolor{gray!6}{0} & \cellcolor{gray!6}{norm\_re} & \cellcolor{gray!6}{full} & \cellcolor{gray!6}{linear\_homo} & \cellcolor{gray!6}{0\_skew} & \cellcolor{gray!6}{linear\_norm} & \cellcolor{gray!6}{time\_seq} & \cellcolor{gray!6}{0.0430000} & \cellcolor{gray!6}{0.1020000} & \cellcolor{gray!6}{0.6490000} & \cellcolor{gray!6}{0.0570000}\\
\addlinespace
136 & High Variance RE & Balanced Groups & norm & linear & 0 & norm\_re & full & linear\_homo & 0\_skew & linear\_norm & time\_seq & 0.0760000 & 0.0850000 & 0.0040000 & 0.0680000\\
\cellcolor{gray!6}{137} & \cellcolor{gray!6}{High Variance Error} & \cellcolor{gray!6}{Unbalanced Groups} & \cellcolor{gray!6}{norm} & \cellcolor{gray!6}{linear} & \cellcolor{gray!6}{0} & \cellcolor{gray!6}{norm\_re} & \cellcolor{gray!6}{full} & \cellcolor{gray!6}{linear\_homo} & \cellcolor{gray!6}{0\_skew} & \cellcolor{gray!6}{linear\_norm} & \cellcolor{gray!6}{time\_seq} & \cellcolor{gray!6}{0.0600000} & \cellcolor{gray!6}{0.1020000} & \cellcolor{gray!6}{0.6950000} & \cellcolor{gray!6}{0.0450000}\\
138 & High Variance RE & Unbalanced Groups & norm & linear & 0 & norm\_re & full & linear\_homo & 0\_skew & linear\_norm & time\_seq & 0.0620000 & 0.1270000 & 0.0030000 & 0.0600000\\
\bottomrule
\end{tabular}}
\end{table}

\begin{center}\includegraphics[width=0.95\linewidth]{Report_draft_files/figure-latex/heter_norm-1} \end{center}

\vspace{-10pt}

\begin{spacing}{0.5}
\begingroup
\fontfamily{ppl}\fontsize{6}{16}\selectfont
Appendix A. The behaviors of residual quantities in Non-Constant Variance and Non-Normality Scenario.
A-1 suggests that non-normality of moderately skewed errors would be well captured by residual plots. Heteroscedasticity in A-2 is more likely to be flagged with high variance errors. A-3 shows deviations of random effects normality induced by heteroscedastic and non-normal errors. 
\endgroup
\end{spacing}

~

\begin{center}\includegraphics[width=0.95\linewidth]{Report_draft_files/figure-latex/heter_lin-1} \end{center}

\vspace{-10pt}

\begin{spacing}{0.5}
\begingroup
\fontfamily{ppl}\fontsize{6}{16}\selectfont
Appendix B. The behaviors of residual quantities in Non-Constant Variance and Non-Linearity Scenario.
Both B-2 and B-4 show direct consequences of non-constant variance and non-linearity. B-1 presents deviations of error term normality caused by misspecification of constant Variance and linearity assumptions. Deviations of random effects normality are apparent only when errors have higher variability in B-3. 
\endgroup
\end{spacing}

~

\begin{center}\includegraphics[width=0.95\linewidth]{Report_draft_files/figure-latex/lin_norm-1} \end{center}

\vspace{-10pt}

\begin{spacing}{0.5}
\begingroup
\fontfamily{ppl}\fontsize{6}{16}\selectfont
Appendix C. The behaviors of residual quantities in Non-Linearity and Non-Normality Scenario.
The skewness of errors is again more well captured than bimodality with low variance errors in C-1. C-2 and C-4 show the rates of heteroscedasticity and non-linearity alleviate as clusters become more unevenly distributed. Highly unbalanced cluster sizes, however, are associated with more severe non-normality of low variance random effects induced by non-linearity and non-normality of errors in C-3. 
\endgroup
\end{spacing}

~

\begin{center}\includegraphics[width=0.95\linewidth]{Report_draft_files/figure-latex/special_case-1} \end{center}

\vspace{-10pt}

\begin{spacing}{0.5}
\begingroup
\fontfamily{ppl}\fontsize{6}{16}\selectfont
Appendix D. Behaviors of residual quantities when misspecifying autocorrelated errors, time variable and only random intercept.
From D-1 and D-4, missing the time variable of longitudinal settings can induce problems of error term normality and linearity when random effects variances are higher. In D-2, higher chance of heteroscedasticity is detected with misspecified autocorrelated errors. Violations to random effect normality occur more frequently with unbalanced cluster sizes in D-3. 
\endgroup
\end{spacing}

~

\newpage

\bibliographystyle{agsm}
\bibliography{LMEsim.bib}

\end{document}
